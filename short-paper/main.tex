\documentclass[a4paper, 14pt]{extarticle}

\usepackage[english]{babel}
\usepackage[utf8]{inputenc}
\usepackage[T1]{fontenc}
\usepackage{amsfonts, geometry, parskip, hyperref}
\geometry{
    paperheight = 11in,
    paperwidth  = 8.5in,
    left   = 1.0in,
    right  = 1.0in,
    top    = 1.0in,
    bottom = 1.0in
}

\begin{document}

\begin{center}
    {\Large\bf Partially Cyclic Public Keys on the Rainbow Signature Scheme} \vspace{.75cm}
    
    Gustavo Zambonin$^{1}$, Ricardo Custódio$^{1}$
    
    $^{1}$Departamento de Informática e Estatística, Universidade Federal de Santa Catarina \\
    88040-900, Florianópolis, Brazil \\
    \texttt{gustavo.zambonin@posgrad.ufsc.br,ricardo.custodio@ufsc.br}
    
     Written for the INE410111 class (Research Methodology in Computer Science), based on~\cite{Petzoldt:phd:2013:jul}. \textbf{This is not a real paper.} \vspace{.75cm}
\end{center}

\textbf{Abstract.} Cryptography based on multivariate equations are one of main the approaches for creating algorithms that are quantum-resistant. Nonetheless, digital signature schemes based on these concepts feature impractical key pair sizes, orders of magnitude greater than commonly-used schemes. We identify a special structure on components of the Rainbow signature scheme that can be used to create a key with a partially cyclic construction, reducing its storage requirements by up to approximately a factor of three.

\textbf{Keywords.} multivariate public key cryptosystem, post-quantum cryptography, digital signature scheme, Rainbow

\section{Introduction}\label{sec:intro}

Security in communications is necessary to ensure trust between parties. For example, a handwritten signature may be placed on a document to attest that the signer is in agreement with its contents. This is no exception when considering digital transit of data, like electronic mail or messages between devices. In this context, cryptographic techniques, such as digital signature schemes, must be employed to ensure desirable properties, namely authenticity, integrity and non-repudiation. These mathematical frameworks permit a signer to hold a private key and a correspondent public key, uniquely related in the sense that one can decode only what its dual has encoded. 

Furthermore, to prevent forgery of signatures by a malicious actor, schemes are naturally bound to computational problems that are virtually unsolvable without the private key. Integer factorisation and discrete logarithm are the most common examples, respectively related to the RSA and ECDSA schemes. Yet, such problems are provably computable in polynomial time by a quantum computer~\cite{Shor:article:1997:oct}. Ergo, post-quantum digital signature schemes are paramount in maintaining secure inter-communications.

One of the approaches for constructing post-quantum schemes is based on systems of linear equations with multiple variables, and is appropriately called multivariate cryptography. Performing computations with these equations is generally very efficient, suitable for performance- or energy-constrained devices. Still, considering common key sizes of 512 bytes for ECDSA, the size of keys for multivariate schemes is prohibitive, as seen in~\cite[Table 6.4]{Petzoldt:phd:2013:jul}. Naturally, it is imperative to research strategies with the intent of reducing these keys.

We focus on reducing the public key size of the Rainbow scheme~\cite{Ding:inproc:2005:jun}, a generalised version of the Unbalanced Oil and Vinegar (UOV) scheme~\cite{Kipnis:inproc:1999:apr} that reduces key sizes and signature length. We apply the strategy described in~\cite[Chapter 7]{Petzoldt:phd:2013:jul}, namely choosing part of the public key instead of a central map, and generating remaining elements as needed.

\emph{Related works.} Most authors focus on reducing the private key size, as seen in the works of Yasuda \emph{et al.}~\cite{Yasuda:inproc:2013:may, Yasuda:inproc:2014:apr, Yasuda:article:2014:sep} and, more recently, Peng and Tang~\cite{Peng:article:2017:jun}, through the use of clever matrix representations. While these approaches cut up to $80\%$ of the private key, we note that public keys are (evidently) more frequently distributed and stored throughout various devices, creating additional limitations for devices that sign or verify messages with various public keys.

\emph{Organization.} In Section~\ref{sec:rainbow}, a straightforward description of the Rainbow signature scheme is given. Afterwards, the rationale behind generating a public key with a special structure is explained in Section~\ref{sec:cyclic}. A comparison between practical instances of the new and old schemes is given in Section~\ref{sec:effect}. Finally, Section~\ref{sec:conclusion} finishes the paper.

\section{Classical Rainbow Signature Scheme}\label{sec:rainbow}

We describe below the Rainbow digital signature scheme due to Ding and Schmidt~\cite{Ding:inproc:2005:jun}, that consists of several layers of ``oil and vinegar''. Consider a finite field $\mathbb{F}_{q}$, and a sequence of integers $0 = v_{0} < v_{1} < v_{2} < \dots < v_{u+1} = n$, with $m = n - v_{1}$. For $i = \{1, \dots, u\}$, let the vinegar variables be $V_{i} = \{1, \dots, v_{i}\}$ and oil variables be $O_{i} = \{v_{i} + 1, \dots, v_{i+1}\}$. Note that when $u = 1$, we get the UOV scheme.

\emph{Key generation.} We randomly generate linear or affine maps $\mathcal{S}: \mathbb{F}_{q}^{m} \rightarrow \mathbb{F}_{q}^{m}$ and $\mathcal{T}: \mathbb{F}_{q}^{n} \rightarrow \mathbb{F}_{q}^{n}$. The central map $\mathcal{F}: \mathbb{F}_{q}^{m} \rightarrow \mathbb{F}_{q}^{n}$
has a special structure. It consists of $m$ quadratic polynomials of the form
$$f^{(k)} = \sum_{i, j \in V_{\ell}, i \leq j} a_{ij}^{(k)} x_{i} x_{j}
        + \sum_{i \in V_{\ell}, j \in O_{\ell}} b_{ij}^{(k)} x_{i} x_{j}
        + \sum_{i \in V_{\ell} \cup O_{\ell}} c_{i}^{(k)} x_{i} + d^{(k)},$$
with $k \in \{v_{1} + 1, \dots, n\}$, where $\ell$ is the only integer such that $k \in O_{\ell}$. The private key is the triple $(\mathcal{S}, \mathcal{F}, \mathcal{T})$. We hide the structure of $\mathcal{F}$ in the public key through its composition with the other maps. Hence, the public key is the map $\mathcal{P} = \mathcal{S} \circ \mathcal{F} \circ \mathcal{T} : \mathbb{F}_{q}^{n} \rightarrow \mathbb{F}_{q}^{m}$.

\emph{Signature generation.} To sign a message $\alpha$, consider a cryptographic hash function $\mathcal{H} : \{0, 1\}^{*} \rightarrow \mathbb{F}_{q}^{m}$, and obtain the digest $d = \mathcal{H}(\alpha)$. Compute $x = \mathcal{S}^{-1}(d)$. 


To invert $\mathcal{F}$, every UOV layer must be inverted recursively. Randomly choose values of $x_{1}, \dots, x_{v_{1}}$ and substitute in the polynomials accordingly. Due to the construction of $\mathcal{F}$, by assigning values to the vinegar variables, the first will obtain a system of $|O_{1}|$ linear equations, solvable through Gaussian elimination.

Finally, compute $\sigma = \mathcal{T}^{-1}(y)$.

\emph{Signature verification.} To verify a signature, compute $d' = \mathcal{P}(\sigma)$. If $d = d'$, then the signature is accepted, and otherwise rejected.

\section{Obtaining a Cyclical Structure}\label{sec:cyclic}

\subsection{The New Scheme, CyclicRainbow}

\emph{Security.}

\section{Effect of the Construction}\label{sec:effect}

\section{Conclusion}\label{sec:conclusion}

\bibliographystyle{alpha}
{\footnotesize
\bibliography{ref}}

\end{document}