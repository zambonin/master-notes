\documentclass[a4paper, 14pt]{extarticle}
\usepackage[utf8]{inputenc}
\usepackage{geometry, parskip, hyperref}
\geometry{
    paperheight = 11in,
    paperwidth  = 8.5in,
    left   = 1.0in,
    right  = 1.0in,
    top    = 1.0in,
    bottom = 1.0in
}

\title{cyclic-rainbow}
\author{Gustavo Zambonin}
\date{October 2018}

\begin{document}

\begin{center}
    {\Large\bf CyclicRainbow -- A Multivariate Signature 
    Scheme with a Partially Cyclic Public Key} \vspace{.75cm}
    
    Gustavo Zambonin$^{1}$, Ricardo Custódio$^{1}$
    
    $^{1}$Departamento de Informática e Estatística, Universidade Federal de Santa Catarina \\
    88040-900, Florianópolis, Brazil \\
    \texttt{gustavo.zambonin@posgrad.ufsc.br,ricardo.custodio@ufsc.br}
    
     Written for the INE410111 class (Research Methodology in Computer Science), based on~\cite{petzoldt2013selecting}.  \vspace{.75cm}
    
    \textbf{Abstract.} 
    
    \textbf{Keywords.} multivariate public key cryptosystem, post-quantum cryptography, digital signature scheme, Rainbow
\end{center}

\section{Introduction}

\emph{Related works.}

\emph{Notation.}

\section{Classical Rainbow Signature Scheme}

\section{Obtaining a Cyclical Structure}

\subsection{The New Scheme, CyclicRainbow}

\emph{Security.}

\section{Effect of the Construction}

\section{Conclusion}

\bibliographystyle{alpha}
{\small
\bibliography{ref}}

\end{document}