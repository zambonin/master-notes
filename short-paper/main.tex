\documentclass[a4paper, 14pt]{extarticle}
\usepackage[utf8]{inputenc}
\usepackage{geometry, parskip, hyperref}
\geometry{
    paperheight = 11in,
    paperwidth  = 8.5in,
    left   = 1.0in,
    right  = 1.0in,
    top    = 1.0in,
    bottom = 1.0in
}

\title{cyclic-rainbow}
\author{Gustavo Zambonin}
\date{October 2018}

\begin{document}

\begin{center}
    {\Large\bf Partially Cyclic Public Keys on the Rainbow Signature Scheme} \vspace{.75cm}
    
    Gustavo Zambonin$^{1}$, Ricardo Custódio$^{1}$
    
    $^{1}$Departamento de Informática e Estatística, Universidade Federal de Santa Catarina \\
    88040-900, Florianópolis, Brazil \\
    \texttt{gustavo.zambonin@posgrad.ufsc.br,ricardo.custodio@ufsc.br}
    
     Written for the INE410111 class (Research Methodology in Computer Science), based on~\cite{petzoldt2013selecting}. \textbf{This is not a real paper.} \vspace{.75cm}
\end{center}

\textbf{Abstract.} Cryptography based on multivariate equations are one of main the approaches for creating algorithms that are quantum-resistant. Nonetheless, digital signature schemes based on these concepts feature impractical key pair sizes, orders of magnitude greater than commonly-used schemes. We identify a special structure on the public key of the Rainbow signature scheme. This strategy can be used to create a key with a partially cyclic construction, reducing its storage requirements by up to a factor of three.

\textbf{Keywords.} multivariate public key cryptosystem, post-quantum cryptography, digital signature scheme, Rainbow

\section{Introduction}

Security in communications is necessary to ensure trust between any number of parties. This is no exception when considering forms of digital exchange of data, such as electronic mail or messages between devices. In this context, cryptographic techniques must be employed

\emph{Related works.}

\emph{Notation.}

\section{Classical Rainbow Signature Scheme}

\section{Obtaining a Cyclical Structure}

\subsection{The New Scheme, CyclicRainbow}

\emph{Security.}

\section{Effect of the Construction}

\section{Conclusion}

\bibliographystyle{alpha}
{\small
\bibliography{ref}}

\end{document}