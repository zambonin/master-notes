\documentclass[10pt]{article}

\usepackage{sbc-template}

\usepackage[english]{babel}
\usepackage[utf8]{inputenc}
\usepackage[T1]{fontenc}
\usepackage{amsfonts, amsmath, array, multirow, hyperref}

\sloppy

\title{Reduction of Key Sizes on Rainbow-like Multivariate Signature Schemes\footnote{
    As submitted to the INE410111 class (Research Methodology in Computer Science).}}

\author{Gustavo Zambonin\inst{1}}

\address{Departamento de Informática e Estatística \\
  Universidade Federal de Santa Catarina \\
  88040-900, Florianópolis, Brazil
  \email{gustavo.zambonin@posgrad.ufsc.br}
}

\begin{document} 

\maketitle

\section{Research Subject}

The amount of data transmitted through digital means has grown exponentially in the last decades. This is related to the achieved heterogeneity of devices that are able to communicate with each other, and to advances in engineering that allow the reduction of size and cost of such contraptions, as well as progress in wireless transmission. Still, the matter of trusting data received from another, possibly unknown device, is highly important. Cryptography in the form of digital signatures is employed to solve matters of authenticity, integrity and non-repudiation of messages. Protecting these pieces of information with such a mathematical framework is enough to prevent malicious actors from forging a message or reading private matters.

Nonetheless, current digital signature schemes are proven to be secure only under the assumption that there are no quantum adversaries. This is the case because quantum computers can solve problems such as integer factorisation or discrete logarithm efficiently~\cite{}, unlike classical computers. Ergo, it is imperative to study cryptography that is quantum-resistant, \emph{i.e.} post-quantum, even if quantum computers are still only of theoretical threat. One of the main approaches for instantiating post-quantum digital signature schemes is based on multivariate quadratic equations. The problems upon which these are based are polynomial system solving and isomorphism of polynomials, not known to be solved more efficiently with a quantum computer~\cite{}. Ergo, these schemes appear to be good candidates for replacing currently used ones.

Multivariate cryptography provides a variety of signature schemes families, such as those based on finite field extensions (HFE and its variations~\cite{}) and those restricted to a single finite field (most schemes based on the Oil--Vinegar principle, such as UOV~\cite{} and Rainbow~\cite{}). All multivariate schemes are extremely efficient when signing and verifying messages~\cite{}, since most computations are based on finite field arithmetic. Still, the Rainbow single field scheme presents the most balanced signature and key sizes with respect to security parameters~\cite{}, as well as being easier to describe and implement accurately. This is important when considering, for instance, limited environments such as smart cards and embedded devices, that still need secure communications but are restrained with regards to memory and processing power, when operating a signed message.

Nevertheless, while signature sizes for Rainbow are already excellent, key sizes are orders of magnitude greater than currently used schemes (\emph{e.g.} RSA has 512 bytes keys). This may be addressed by means of employing special structures into its public and/or private keys. Since these can be stored as matrices, sparseness and cyclic approaches are common strategies, but it is important to note that the restriction of key pairs to matrices with these characteristics may negatively impact the overall security of the scheme. Thus, we will focus on the creation of secure methods for the reduction of public and private keys on the Rainbow signature scheme.

\section{Related Works}

There has been a considerable amount of work done when dealing with reduction of private and public keys on the Rainbow signature scheme. Still, not all of the published works maintain the security \emph{status quo} of their common predecessor. For instance, a scheme based on non-commutative rings due to Yasuda \emph{et al.}~\cite{} reduces private keys up to a factor of four, but was subsequently analysed and broken by independent researchers~\cite{}. Another example is that of a scheme based on circulant matrices due to Peng and Tang~\cite{}, that reduces private key size by as much as $45\%$ and gives secure parameters for another work of Yasuda \emph{et al.}~\cite{}, but it was also deemed insecure by Hashimoto~\cite{}. 

Still, there are various examples of schemes that resist currently known cryptanalysis. Petzoldt \emph{et al.} suggest a scheme based on cyclic matrices~\cite{} that reduces the public key size by as much as $62\%$. A scheme by Yasuda \emph{et al.}~\cite{} uses sparse matrices to reduce private key sizes by up to $76\%$. Further, we point out works that deal with the special case of Rainbow known as UOV. Szepieniec \emph{et al.}~\cite{} reduces the public key while increasing the signature size; Beullens and Preneel~\cite{} lift the public and central maps of the scheme to an extension field, reducing the public key size by an order of magnitude while increasing the signature size; and Petzoldt \emph{et al.}~\cite{} use linear recurring sequences to reduce the public key size by a factor of 7.5. Generalisation of these approaches for Rainbow is an open problem in all works.

\section{Objectives}

\subsection{Hypothesis}

\section{Methodology}

\subsection{Expected Results}

\bibliographystyle{sbc}
\bibliography{sbc-template}

\end{document}
