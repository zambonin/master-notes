\documentclass[10pt]{article}

\usepackage{sbc-template}

\usepackage[english]{babel}
\usepackage[utf8]{inputenc}
\usepackage[T1]{fontenc}
\usepackage{amsfonts, amsmath, array, multirow, hyperref}

\sloppy

\title{Reduction of Key Sizes on Rainbow-like Multivariate Signature Schemes\footnote{
    As submitted to the INE410111 class (Research Methodology in Computer Science).}}

\author{Gustavo Zambonin\inst{1}}

\address{Departamento de Informática e Estatística \\
  Universidade Federal de Santa Catarina \\
  88040-900, Florianópolis, Brazil
  \email{gustavo.zambonin@posgrad.ufsc.br}
}

\begin{document} 

\maketitle

\section{Research Subject}

The amount of data transmitted through digital means has grown exponentially in the last decades. This is related to the achieved heterogeneity of devices that are able to communicate with each other, and to advances in engineering that allow the reduction of size and cost of such contraptions, as well as progress in wireless transmission. Still, the matter of trusting data received from another, possibly unknown device, is highly important. Cryptography in the form of digital signatures is employed to solve matters of authenticity, integrity and non-repudiation of messages. Protecting these pieces of information with such a mathematical framework is enough to prevent malicious actors from forging a message or reading private matters.

Nonetheless, current digital signature schemes are proven to be secure only under the assumption that there are no quantum adversaries. This is the case because quantum computers can solve problems such as integer factorization or discrete logarithm efficiently~\cite{}, unlike classical computers. Ergo, it is imperative to build schemes that are 

\section{Related Works}

\section{Objectives}

\subsection{Hypothesis}

\section{Methodology}

\subsection{Expected Results}

\bibliographystyle{sbc}
\bibliography{sbc-template}

\end{document}
