\documentclass[10pt]{article}

\usepackage{sbc-template}

\usepackage[english]{babel}
\usepackage[utf8]{inputenc}
\usepackage[T1]{fontenc}
\usepackage{enumitem}

\sloppy

\title{Reduction of Key Sizes on Rainbow-like Multivariate Signature Schemes\footnote{
    As submitted to the INE410111 class (Research Methodology in Computer Science).}}

\author{Gustavo Zambonin\inst{1}}

\address{Departamento de Informática e Estatística \\
  Universidade Federal de Santa Catarina \\
  88040-900, Florianópolis, Brazil
  \email{gustavo.zambonin@posgrad.ufsc.br}}

\begin{document} 

\maketitle

\section{Research Subject}

The amount of data transmitted through digital means has grown exponentially in the last decades. Advances in engineering have provided smaller and cheaper devices, usually integrated with wireless capabilities, facilitating heterogeneous communications. Still, the matter of trusting data received from another, possibly unknown machine, is highly important. Digital signatures are cryptographic frameworks employed to solve matters of authenticity, integrity and non-repudiation of messages. Protecting information with this method is sufficient to prevent malicious actors from forging a message or reading private matters, according to Goldreich~\cite{Goldreich:book:2004}.

Nonetheless, current digital signature schemes are proven to be secure only under the assumption that there are no quantum adversaries. Unlike classical computers, quantum machines can execute Shor's algorithm~\cite{Shor:article:1997:oct} to efficiently solve problems such as integer factorisation or discrete logarithm. Ergo, it is imperative to study cryptography that is quantum-resistant, \emph{i.e.} post-quantum, even if classical machines are still the norm. One of the main approaches for instantiating post-quantum digital signature schemes is based on multivariate quadratic equations. The problems upon which these are based are polynomial system solving and isomorphism of polynomials, not known to be solved more efficiently with a quantum computer~\cite{Bernstein:book:2008}. Ergo, these schemes appear to be good candidates for replacing currently used ones.

Multivariate cryptography provides a variety of signature schemes families, such as those based on finite field extensions (HFE and its variations~\cite{Patarin:inproc:1996:may}) and those restricted to a single finite field (most schemes based on the Oil--Vinegar principle, such as UOV~\cite{Kipnis:inproc:1999:apr} and Rainbow~\cite{Ding:inproc:2005:jun}). All multivariate schemes are extremely efficient when signing and verifying messages~\cite{Ding:book:2006}, since most computations are based on finite field arithmetic. Still, the Rainbow single field scheme presents the most balanced signature and key sizes with respect to security parameters~\cite{Ding:article:2017:jul}, as well as being easier to describe and implement accurately. This is important when considering, for instance, limited environments such as smart cards and embedded devices, that still need secure communications but are restrained with regards to memory and processing power, when operating a signed message.

Nevertheless, while signature sizes for Rainbow are already excellent, key sizes are orders of magnitude greater than currently used schemes (\emph{e.g.} RSA has 512 bytes keys). This may be addressed by means of employing special structures into its public and private keys. Since these can be stored as matrices, sparseness and cyclic approaches are common strategies, but it is important to note that the restriction of key pairs to matrices with these characteristics may negatively impact the overall security of the scheme. Thus, we will focus on the creation of secure methods for the reduction of public and private keys on the Rainbow signature scheme.

\section{Related Works}

There has been a considerable amount of work done when dealing with reduction of private and public keys on the Rainbow signature scheme. Still, not all of the published works maintain the security of their common predecessor. For instance, a scheme based on non-commutative rings due to Yasuda \emph{et al.}~\cite{Yasuda:inproc:2012:feb} reduces private keys up to a factor of four, but was subsequently analysed and broken by independent researchers~\cite{Hashimoto:inproc:2013:feb,Thomae:inproc:2012:sep}. Another example is that of a scheme based on circulant matrices due to Peng and Tang~\cite{Peng:article:2017:jun}, that reduces private key size by as much as $45\%$ and gives secure parameters for other works of Yasuda \emph{et al.}~\cite{Yasuda:inproc:2013:may,Yasuda:inproc:2014:apr}, but it was also deemed insecure by Hashimoto~\cite{Hashimoto:misc:2018:oct}. 

Still, there are various examples of schemes that resist currently known cryptanalysis. Petzoldt \emph{et al.} suggest a scheme based on cyclic matrices~\cite{Petzoldt:inproc:2010:dec} that reduces the public key size by as much as $62\%$. A scheme due to Yasuda \emph{et al.}~\cite{Yasuda:article:2014:sep} uses sparse matrices to reduce private key sizes by up to $76\%$. The work of Petzoldt~\cite{Petzoldt:phd:2013:jul} lists various possible approaches for reduction of public keys, pointing out that some elements in the coefficient matrix of the public key may be fixed without affecting the underlying security of the scheme. 

Moreover, there are also works that deal with the special case of Rainbow known as UOV. Szepieniec \emph{et al.}~\cite{Szepieniec:inproc:2017:jun} reduces the public key while increasing the signature size; Beullens and Preneel~\cite{Beullens:inproc:2017:dec} lift the public and central maps of the scheme to an extension field, reducing the public key size by an order of magnitude while increasing the signature size; and Petzoldt and Bulygin~\cite{Petzoldt:inproc:2012:nov} use linear recurring sequences to reduce the public key size by a factor of 7.5. Generalisation of these approaches for Rainbow is an open problem in all works. Furthermore, none of the aforementioned works deal with reduction of both keys in the key pair.

\section{Hypothesis}

It is evident that no works in the literature have reduced both private and public keys on the Rainbow signature scheme. Thus, our target research question asks if there are any restrictions in doing so. It is known that public key cryptography features a duality in the sense that codifications done by a private key will only be undone by its corresponding public key, and vice-versa, that is, both keys are intrinsically related. The usual procedure is to use a seeded pseudo-random number generator to create a private key, and derive the public key from that. In the case of Rainbow, we wish to know if this method can be executed using only matrices with a desirable symmetry that may act as a key pair of this scheme. 

\section{Objectives}

With this project, we aim to reduce the size of both private and public keys on the Rainbow signature scheme. This will be done through the introduction of a special structure into the matrix representation of both keys, that will allow a compact presentation and decrease overall storage requirements for the scheme. Based on this, the following specific objectives are presented.

\begin{enumerate}[label=\alph*.]
    \item Establish matrix symmetries that are fit to be used in the context of multivariate public key cryptography, that is, observe if public keys generated from specially composed private keys will maintain their underlying structure;
    \item Verify whether these matrix structures (\emph{e.g.} Hessenberg, Hankel, centrosymmetric) may be used to generate key pairs and their direct effect on the security of the scheme;
    \item Development of a Rainbow signature scheme variant employing a matrix structure with the intent of reducing private and public key sizes;
    \item Provide sets of parameters reasonable for environments with distinct security requirements, for instance, desktop computers, mobile and embedded devices, servers etc..
\end{enumerate}

\section{Methodology}

We will demonstrate the previously mentioned objectives through a chronological list of activities presented below. These partial endeavours will enhance our knowledge in special matrix structures, the Oil--Vinegar principle, Rainbow signature scheme and its modern variants, as well as new cryptanalysis methods. Furthermore, note that there are some specific literary and oral productions that must be created, as part of the master's program requirements.

\begin{enumerate}[label=\alph*.]
    \item Bibliographic review and maintenance of a database with works that are related to the Rainbow signature scheme, variations focusing on the reduction of public or private key sizes, and related cryptanalysis;
    \item Deep study of matrix-like structures that may be represented in a compact form;
    \item Creation of an algorithm that can generate a compact private-public key pair;
    \item Procedure to check if the reduced key space of compact key pairs is still large enough to prevent brute-force search;
    \item Apply currently known cryptanalytic methods to the new Rainbow-like signature scheme and compare it with the original method;
    \item Compare the new algorithm with other schemes that also reduce key sizes through experiments;
    \item At least one scientific contribution in the form of a conference paper or journal article;
    \item Oral presentation about the work done so far, that will act as a qualification exam;
    \item Production of a dissertation about the subject;
    \item Oral presentation about the concluded work.
\end{enumerate}

\section{Expected Results}

We wish to collaborate with individuals and institutions alike, with the intent of broadening our knowledge and push the boundaries of optimisations with regards to reducing key sizes in single-field schemes. As such, we expect the publication of scientific material about this subject, collaboration with students, professors, independent and contracted researchers, promotion of our university through oral presentations of this research, a breakthrough result on the relation between compact private and public keys on Rainbow, and free, open-source software implementations of these algorithms.

\bibliographystyle{sbc}
\bibliography{sbc-template}

\end{document}
