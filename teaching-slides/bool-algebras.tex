\documentclass[12pt]{beamer}

\usepackage[brazil]{babel}
\usepackage[T1]{fontenc}
\usepackage[utf8]{inputenc}
\usepackage{enumitem, hyperref, tikz}

\setitemize{
  itemsep=1em,
  label=\usebeamerfont*{itemize item}
    \usebeamercolor[fg]{itemize item}
    \usebeamertemplate{itemize item}
}

\usetikzlibrary{positioning, decorations.markings}
\tikzset{small/.style={draw,fill,circle,inner sep=1pt,outer sep=0pt}}

\setbeamertemplate{footline}[frame number]{}
\setbeamertemplate{navigation symbols}{}

\title{Álgebras Booleanas finitas}
\author{Gustavo Zambonin}
\institute{
  \includegraphics[scale=0.15]{ufsc}                    \\ \vspace{-4mm}
  Universidade Federal de Santa Catarina                \\
  Departamento de Informática e Estatística             \\
  INE5601 --- Fundamentos Matemáticos da Informática    \\ \vspace{2mm}
  \texttt{gustavo.zambonin@posgrad.ufsc.br}
}
\date{}

\begin{document}

\begin{frame}[plain,noframenumbering]
  \titlepage{}
\end{frame}

\begin{frame}
  \frametitle{Contexto}
  \begin{itemize}
    \item<1-> Reticulados
    \begin{itemize}[itemsep=0pt]
      \item \emph{Poset} onde qualquer par de elementos tem supremo e ínfimo
      \item Representado também como estrutura algébrica
    \end{itemize}
    \item<2-> Isomorfismo entre reticulados
    \begin{itemize}[itemsep=0pt]
      \item Função bijetora que mapeia elementos entre dois reticulados
    \end{itemize}
    \item<3-> Tipos de reticulados
    \begin{itemize}[itemsep=0pt]
      \item Limitado, complementado, distributivo
      \item<4->
          \href{https://upload.wikimedia.org/wikipedia/en/c/cc/Lattice_v4.png}{Mapa
            de reticulados}
    \end{itemize}
  \end{itemize}
\end{frame}

\begin{frame}
  \frametitle{Exemplos práticos}
  \begin{itemize}
    \item<1-> Cálculo proposicional pode ser demonstrado logicamente
        equivalente a uma expressão Booleana
    \item<2-> Modelagem de circuitos em engenharia elétrica, para representar
        estados de alta e baixa tensão
    \begin{itemize}
      \item Criação de portas lógicas (AND, OR, NAND, NOR, XOR, XNOR)
    \end{itemize}
    \item<3-> Construção de caixas de substituição em criptografia simétrica,
        com funções Booleanas
    \item<4-> Ideia geral: formalismo para descrever operações lógicas
  \end{itemize}
\end{frame}

\begin{frame}
  \frametitle{Reticulados de conjuntos sob inclusão}
  \begin{itemize}
    \item Considere um \emph{poset} $(\mathcal{P}(S), \subseteq)$, onde $S$ é
        finito
    \begin{itemize}[itemsep=0pt]
      \item $\forall t_1, t_2 \in \mathcal{P}(S)$, \\
          $\inf(\{t_1, t_2\}) = t_1 \cap t_2, \quad
           \sup(\{t_1, t_2\}) = t_1 \cup t_2$
    \end{itemize}
    \item<2-> Tome $S_{1} = \{x_{1}, \dots, x_{n}\}, S_{2} = \{y_{1}, \dots,
        y_{n}\}$
    \begin{itemize}[itemsep=0pt]
      \item Existe um isomorfismo $f$ que mapeia $x_{i} \rightarrow y_{i}, i
          \in \{1, \dots, n\}$
      \item Para quaisquer subconjuntos $A, B \subseteq S$, então $A \subseteq
          B \Leftrightarrow f(A) \subseteq f(B)$
    \end{itemize}
    \item<3-> Então, $(\mathcal{P}(S_{1}), \subseteq)$ e $(\mathcal{P}(S_{2}),
        \subseteq)$ são isomórficos
  \end{itemize}
\end{frame}

\begin{frame}
  \frametitle{Exemplo}
  \framesubtitle{Isomorfismo entre reticulados}
  \begin{figure}
  \resizebox{10cm}{!}{
    \begin{tikzpicture}
      \node [small, label=above:{\tiny $\{2, 3, 5\}$}] (xyz) at (0, 0) {};
      \node [small, below left=of xyz, label=left:{\tiny $\{2, 3\}$}] (xy) {};
      \node [small, below right=of xyz, label=right:{\tiny $\{3, 5\}$}] (yz) {};
      \node [small, below=of xyz, label=right:{\tiny $\{2, 5\}$}] (xz) {};
      \node [small, below=of xy, label=left:{\tiny $\{2\}$}] (x) {};
      \node [small, below=of xz, label=right:{\tiny $\{3\}$}] (y) {};
      \node [small, below=of yz, label=right:{\tiny $\{5\}$}] (z) {};
      \node [small, below=of y, label=below:{\tiny $\emptyset$}] (0) {};
      \draw (0) edge (x)  (0)  edge (y)   (0)  edge (z)   (x)  edge (xy)
            (x) edge (xz) (y)  edge (xy)  (y)  edge (yz)  (z)  edge (xz)
            (z) edge (yz) (xy) edge (xyz) (xz) edge (xyz) (yz) edge (xyz);
    \end{tikzpicture}
    \quad
    \begin{tikzpicture}
      \node [small, label=above:{\tiny $\{x, y, z\}$}] (xyz) at (0, 0) {};
      \node [small, below left=of xyz, label=left:{\tiny $\{x, y\}$}] (xy) {};
      \node [small, below right=of xyz, label=right:{\tiny $\{y, z\}$}] (yz) {};
      \node [small, below=of xyz, label=right:{\tiny $\{x, z\}$}] (xz) {};
      \node [small, below=of xy, label=left:{\tiny $\{x\}$}] (x) {};
      \node [small, below=of xz, label=right:{\tiny $\{y\}$}] (y) {};
      \node [small, below=of yz, label=right:{\tiny $\{z\}$}] (z) {};
      \node [small, below=of y, label=below:{\tiny $\emptyset$}] (0) {};
      \draw (0) edge (x)  (0)  edge (y)   (0)  edge (z)   (x)  edge (xy)
            (x) edge (xz) (y)  edge (xy)  (y)  edge (yz)  (z)  edge (xz)
            (z) edge (yz) (xy) edge (xyz) (xz) edge (xyz) (yz) edge (xyz);
    \end{tikzpicture}
  }
  \end{figure}
  \begin{itemize}[itemsep=0pt]
    \item Seja $S_{1} = \{2, 3, 5\}, S_{2} = \{x, y, z\}$, então $n = 3$
    \item<2-> $f : S_{1} \rightarrow S_{2}$ \\
        $f(\emptyset) = \emptyset,\; f(\{2\}) = \{x\},\; f(\{3\}) = \{y\},\;
          \dots$
  \end{itemize}
\end{frame}

\begin{frame}
  \frametitle{Reticulado $(\mathcal{P}(S), \subseteq)$ genérico}
  \begin{itemize}
    \item Portanto, $(\mathcal{P}(S), \subseteq)$ independe de $S$
    \item Reticulado determinado apenas por $n = |S|$
    \begin{itemize}
      \item O número de elementos do reticulado sempre será da forma
          $|\mathcal{P}(S)| = 2^{n}$
    \end{itemize}
    \item<2-> É possível construir um reticulado genérico, composto de
        $n$-tuplas de 0 e 1, chamado $B_{n}$
    \begin{itemize}
      \item 0 denota a ausência do elemento no subconjunto, e 1 a presença
    \end{itemize}
  \end{itemize}
\end{frame}

\begin{frame}
  \frametitle{Ordenamento em $B_{n}$}
  \begin{itemize}
    \item<1-> Tome $x = x_{1} x_{2} \dots x_{n}, y = y_{1} y_{2} \dots y_{n}
        \in B_{n}$ quaisquer
    \begin{itemize}[itemsep=0pt]
      \item<2-> Ordenação ``lexicográfica'': $x \preccurlyeq y \Leftrightarrow
          a_{k} \preccurlyeq b_{k}, k \in \{1, \dots, n\}$
      \item<3-> $x \wedge y = s_{1} s_{2} \dots s_{n}, \quad s_{k} =
          \min{(a_{k}, b_{k})}$
      \item<4-> $x \vee y = z_{1} z_{2} \dots z_{n}, \quad z_{k} =
          \max{(a_{k}, b_{k})}$
      \item<5-> Complemento: se $z_{k} = 1, z'_{k} = 0$ e vice-versa
    \end{itemize}
    \item<6-> Note que $(B_{n}, \preccurlyeq)$ é isomórfico a $(\mathcal{P}(S),
        \subseteq)$
  \end{itemize}
\end{frame}

\begin{frame}
  \frametitle{Ordenamento em $B_{n}$}
  \begin{itemize}
    \item<1-> Então, existe uma correspondência entre reticulados sob conjuntos
        e $B_{n}$, da seguinte forma
    \item Para quaisquer subconjuntos $A, B \in \mathcal{P}(S)$
    \begin{itemize}[itemsep=0pt]
      \item<2-> $x \preccurlyeq y \Leftrightarrow A \subseteq B$
      \item<3-> $x \wedge y \Leftrightarrow A \cap B$
      \item<4-> $x \vee y \Leftrightarrow A \cup B$
      \item<5-> $x' \Leftrightarrow \overline{A}$
    \end{itemize}
  \end{itemize}
\end{frame}

\begin{frame}
  \frametitle{Álgebras Booleanas finitas}
  \begin{itemize}
    \item Um reticulado complementado distributivo é chamado de
        \textbf{álgebra Booleana}
    \item De maneira equivalente, um reticulado finito isomórfico a $B_{n}$ é
        uma álgebra Booleana
    \item Note que é possível representar quaisquer reticulados
        $(\mathcal{P}(S), \subseteq)$ como $B_{|S|}$
    \begin{itemize}
      \item Ou seja, todo reticulado isomórfico a $(\mathcal{P}(S),
          \subseteq)$ também é uma álgebra Booleana
    \end{itemize}
  \end{itemize}
\end{frame}

\begin{frame}
  \frametitle{Exemplo}
  \framesubtitle{Álgebras Booleanas finitas}
  \vspace{-1cm}
  \begin{figure}
    \begin{tikzpicture}
      \node [small, label=below:{\tiny \color{white}{0}}] (xyz) at (0, 0) {};
    \end{tikzpicture}
    \qquad
    \begin{tikzpicture}
      \node [small, label=above:{\tiny $1$}] (xyz) at (0, 0) {};
      \node [small, below=of xyz, label=below:{\tiny $0$}] (xz) {};
      \draw (xz) edge (xyz);
    \end{tikzpicture}
    \qquad
    \begin{tikzpicture}
      \node [small, label=above:{\tiny $11$}] (xyz) at (0, 0) {};
      \node [small, below left=of xyz, label=left:{\tiny $01$}] (xy) {};
      \node [small, below right=of xyz, label=right:{\tiny $10$}] (yz) {};
      \node [small, below=of xz, label=below:{\tiny $0$}] (y) {};
      \draw (y)  edge (xy)  (y)  edge (yz)  (xy) edge (xyz)  (yz) edge (xyz);
    \end{tikzpicture}
    \qquad
    \begin{tikzpicture}
      \node [small, label=above:{\tiny $111$}] (xyz) at (0, 0) {};
      \node [small, below left=of xyz, label=left:{\tiny $110$}] (xy) {};
      \node [small, below right=of xyz, label=right:{\tiny $011$}] (yz) {};
      \node [small, below=of xyz, label=right:{\tiny $101$}] (xz) {};
      \node [small, below=of xy, label=left:{\tiny $100$}] (x) {};
      \node [small, below=of xz, label=right:{\tiny $010$}] (y) {};
      \node [small, below=of yz, label=right:{\tiny $001$}] (z) {};
      \node [small, below=of y, label=below:{\tiny $000$}] (0) {};
      \draw (0) edge (x)  (0)  edge (y)   (0)  edge (z)   (x)  edge (xy)
            (x) edge (xz) (y)  edge (xy)  (y)  edge (yz)  (z)  edge (xz)
            (z) edge (yz) (xy) edge (xyz) (xz) edge (xyz) (yz) edge (xyz);
    \end{tikzpicture}
  \end{figure}
  \begin{itemize}
    \item Álgebras Booleanas mais simples: $B_{0}, B_{1}, B_{2}, B_{3}$
    \item Número de elementos: $2^{0} = 1, 2^{1} = 2, 2^{2} = 4, 2^{3} = 8$
  \end{itemize}
\end{frame}

\begin{frame}
  \frametitle{Outros reticulados isomórficos a $B_{n}$}
  \begin{itemize}
    \item<1-> Reticulados que não são da forma $(\mathcal{P}(S), \subseteq)$
        também podem ser álgebras Booleanas
    \item<2-> Considere o reticulado $D_{n}$, onde $S$ é composto pelos
        divisores de $n$ e a relação parcial é de divisibilidade
    \begin{itemize}
      \item $D_{30} = (\{1, 2, 3, 5, 6, 10, 15, 30\}, \mid)$
    \end{itemize}
    \item<3-> Observe que $D_{30}$ é isomórfico a $B_{3}$
    \begin{itemize}
      \item<4-> {\small $f(1) = 000, f(2) = 100, f(3) = 010, f(5) = 001, f(6) =
          110, f(10) = 101, f(15) = 011, f(30) = 111$}
    \end{itemize}
  \end{itemize}
\end{frame}

\begin{frame}
  \frametitle{Determinação de álgebras Booleanas}
  \begin{itemize}
    \item<1-> Todo reticulado que não tenha $2^{n}$ elementos não pode ser uma
        álgebra Booleana
    \item<2-> Um reticulado com $2^{n}$ elementos é uma condição necessária, mas
        não suficiente
    \begin{itemize}[itemsep=0pt]
      \item<3-> É necessário demonstrar o isomorfismo com $B_{n}$ ou
          $(\mathcal{P}(S), \subseteq)$
      \item<3-> Comparar o diagrama de Hasse é possível para conjuntos pequenos
    \end{itemize}
  \end{itemize}
\end{frame}

\begin{frame}
  \frametitle{Determinação de álgebras Booleanas}
  \begin{itemize}
    \item<1-> Todo reticulado não limitado não será álgebra Booleana
    \item<2-> Todo elemento deverá ter um único complemento
    \item<3-> O reticulado $D_{p}$, onde $p = p_{1} \times p_{2} \times \dots
        \times p_{k}$, com $p_{1} \neq p_{2} \neq \dots \neq p_{k}$, é álgebra
          Booleana?
    \begin{itemize}[itemsep=0pt]
      \item $S = (\{p_{1}, \dots, p_{k}\})$, então $D_{p} = (S, \mid)$
      \item<4-> Note que existe um isomorfismo $f$, de modo que $\forall T \in
          \mathcal{P}(S),\; f(T) = t_1 \dots t_{k}$
      \item<5-> Portanto, $D_{p}$ é uma álgebra Booleana
    \end{itemize}
  \end{itemize}
\end{frame}

\begin{frame}
  \frametitle{Identidades axiomáticas de álgebras Booleanas}
  \begin{itemize}
    \item<1-> Defina uma álgebra Booleana como $(S, \vee, \wedge, \neg, \bot,
        \top)$
    \begin{itemize}
      \item Ou seja, um conjunto finito $S$, operações binárias de junção e
          encontro, operação unária de complemento, elementos mínimo e máximo
    \end{itemize}
    \item<2-> Então, as leis abaixo são verdade
    \begin{itemize}
      \item Associatividade, comutatividade, absorção, identidade,
          distributividade, complementação
    \end{itemize}
    \item<3-> Note que estes axiomas são derivados das definições de reticulado
        limitado, distributivo e complementado
  \end{itemize}
\end{frame}

\begin{frame}
  \frametitle{Identidades axiomáticas de álgebras Booleanas}
  \begin{itemize}
    \item<1-> Outras três propriedades podem ser derivadas
    \item<1-> Para elementos quaisquer $x, y$ de uma álgebra Booleana
    \begin{itemize}[itemsep=0pt]
      \item<2-> $\neg (\neg x) = x$ (lei da involução)
      \item<3-> $\neg (x \wedge y) = \neg x \vee \neg y$ (lei de De Morgan I)
      \item<3-> $\neg (x \vee y) = \neg x \wedge \neg y$ (lei de De Morgan II)
    \end{itemize}
    \item<4-> Assim como em conjuntos, pois para $S$ qualquer e $A, B \subseteq
        S$ quaisquer
    \begin{itemize}
      \item $\overline{(\overline{A})}, \quad \overline{(A \cap B)} =
          \overline{A} \cup \overline{B}, \quad \overline{(A \cup B)} =
            \overline{A} \cap \overline{B}$
    \end{itemize}
  \end{itemize}
\end{frame}

\begin{frame}
  \frametitle{Material de estudo}
  \bibliographystyle{apalike}
  \bibliography{ref}
  \begin{itemize}[itemsep=0pt]
    \nocite{Kolman:book:1999}
    \item Kolman: leitura das páginas 217--223 (especialmente p. 222) e
        resolução dos exercícios 1--21
    \nocite{Rosen:book:2011}
    \item Rosen: leitura das páginas 811--817 e resolução dos
        exercícios 1--4, 24--28
  \end{itemize}
\end{frame}

\end{document}
