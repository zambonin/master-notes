\documentclass[12pt]{beamer}

\usepackage[brazil]{babel}
\usepackage[T1]{fontenc}
\usepackage[utf8]{inputenc}
\usepackage{enumitem, xcolor, tikz}

\setitemize{
  itemsep=1em,
  label=\usebeamerfont*{itemize item}
    \usebeamercolor[fg]{itemize item}
    \usebeamertemplate{itemize item}
}

\usetikzlibrary{positioning, decorations.markings}
\tikzset{small/.style={draw,fill,circle,inner sep=1pt,outer sep=0pt}}

\setbeamertemplate{footline}[frame number]{}
\setbeamertemplate{navigation symbols}{}

\title{Introdução à teoria de reticulados}
\author{Gustavo Zambonin}
\institute{
  \includegraphics[scale=0.15]{ufsc}                    \\ \vspace{-4mm}
  Universidade Federal de Santa Catarina                \\
  Departamento de Informática e Estatística             \\
  INE5601 --- Fundamentos Matemáticos da Informática    \\ \vspace{2mm}
  \texttt{gustavo.zambonin@posgrad.ufsc.br}
}
\date{}

\begin{document}

\begin{frame}[plain,noframenumbering]
  \titlepage
\end{frame}

\begin{frame}
  \frametitle{Contexto}
  \begin{itemize}
    \item<1-> Conjuntos parcialmente ordenados (\emph{posets})
    \begin{itemize}
      \item Par ordenado de conjunto qualquer com relação reflexiva,
          antissimétrica e transitiva
    \end{itemize}
    \item<2-> Diagramas de Hasse
    \begin{itemize}
      \item Representação gráfica intuitiva de \emph{posets}
    \end{itemize}
    \item<3-> Cota superior, inferior, elementos extremos
    \begin{itemize}
      \item Supremo e ínfimo
    \end{itemize}
  \end{itemize}
\end{frame}

\begin{frame}
  \frametitle{Exemplos práticos}
  \begin{itemize}
    \item Ontologias (representação de entidades e evento de acordo com
        categorias)
    \item Fluxo de informação entre dois processos estocásticos
    \item Descrição de herança múltipla em linguagens de programação orientadas
        a objetos
    \item<2-> Ideia geral: estruturas abstratas que permitem a
        operacionalização de vários elementos em um conjunto
  \end{itemize}
\end{frame}

\begin{frame}
  \frametitle{O termo ``reticulado''}
  \begin{itemize}
    \item Não são relacionados exclusivamente à teoria de ordem
    \item Existem reticulados geométricos (malha de pontos no plano Euclidiano)
    \begin{itemize}
      \item Utilizados em ciência dos materiais e criptografia
    \end{itemize}
    \item<2-> Todo reticulado geométrico pode ser ``convertido'' para uma
        descrição utilizando um \emph{poset}
    \begin{itemize}
      \item O contrário não se aplica
    \end{itemize}
  \end{itemize}
\end{frame}

\begin{frame}
  \frametitle{Notação}
  \begin{itemize}
    \item O supremo de um subconjunto $K$ de um \emph{poset}, $\sup(K)$, é
        também chamado de junção ou \emph{join}, e denotado $\vee K$
    \item O ínfimo de um subconjunto $K$ de um \emph{poset}, $\inf(K)$, é
        também chamado de encontro ou \emph{meet}, e denotado $\wedge K$
    \item Um reticulado também pode ser chamado de \emph{lattice}
  \end{itemize}
\end{frame}

\begin{frame}
  \frametitle{Semirreticulados}
  \begin{itemize}
    \item<1-> Um \emph{poset} onde todos os pares de elementos possuem supremo,
        ou todos possuem ínfimo
    \item<2-> Se todos possuem supremo, é chamado de \textbf{semirreticulado de
        junção}, ou \emph{join-semilattice}
    \item<3-> Se todos possuem ínfimo, é chamado de \textbf{semirreticulado de
        encontro}, ou \emph{meet-semilattice}
    \item<4-> Junção e encontro são, portanto, operações binárias sobre os
        elementos do semirreticulado
  \end{itemize}
\end{frame}

\begin{frame}
  \frametitle{Semirreticulados}
  \begin{itemize}
    \item<1-> Formalmente, dado um \emph{poset} $(S, \preccurlyeq)$ e, $\forall
        s_1, s_2 \in S$
    \begin{itemize}[itemsep=0pt]
      \item O \emph{poset} é \emph{join-semilattice} se $\sup(\{s_1, s_2\})$,
          também denotado $s_1 \vee s_2$
      \item O \emph{poset} é \emph{meet-semilattice} se $\inf(\{s_1, s_2\})$,
          também denotado $s_1 \wedge s_2$
    \end{itemize}
    \item<2-> Note que $s_1 \vee s_2 = s_2$ e $s_1 \wedge s_2 = s_1$, para $s_1
        \preccurlyeq s_2$
    \item<3-> Exemplo clássico: tome um conjunto qualquer $T$
    \begin{itemize}[itemsep=0pt]
      \item O \emph{poset} $(\mathcal{P}(T) \; \backslash \; \emptyset,
          \subseteq)$ é um \emph{join-semilattice}
      \item O \emph{poset} $(\mathcal{P}(T) \; \backslash \; T, \subseteq)$ é
          um \emph{meet-semilattice}
    \end{itemize}
  \end{itemize}
\end{frame}

\begin{frame}
  \frametitle{Reticulados}
  \begin{itemize}
    \item Um \emph{poset} onde qualquer par do conjunto possui um ínfimo e um
        supremo
    \item Formalmente, um \emph{poset} $(S, \preccurlyeq)$ é um
        \textbf{reticulado} ou \emph{lattice} quando, $\forall s_1, s_2 \in S$,
          $\inf(\{s_1, s_2\})$ e $\sup(\{s_1, s_2\})$ existem
    \item De maneira equivalente, um reticulado é um \emph{poset} que, ao
        mesmo tempo, é um \emph{join-semilattice} e \emph{meet-semilattice}
  \end{itemize}
\end{frame}

\begin{frame}
  \frametitle{Exemplo}
  \framesubtitle{Reticulados}
  \begin{itemize}
    \item<1-> Considere o \emph{poset} $(\mathbb{Z}^{+}, \mid)$. Este
        \emph{poset} é um reticulado?
    \begin{itemize}[itemsep=0pt]
      \item<2-> Observe que, para quaisquer $a, b \in \mathbb{Z}^{+}$,
          $\inf(\{a, b\}) = \text{mmc}(a, b)$ e $\sup(\{a, b\}) = \text{mdc}(a,
            b)$
      \item<2-> Portanto, $(\mathbb{Z}^{+}, \mid)$ é um reticulado
    \end{itemize}
    \item<3-> Considere o \emph{poset} $(\mathcal{P}(T), \subseteq)$. Este
        \emph{poset} é um reticulado?
    \begin{itemize}[itemsep=0pt]
      \item<4-> Observe que, para quaisquer $t_1, t_2 \in \mathcal{P}(T)$,
          $\inf(\{t_1, t_2\}) = t_1 \cap t_2$ e $\sup(\{t_1, t_2\}) = t_1 \cup
            t_2$
      \item<4-> Portanto, $(\mathcal{P}(T), \subseteq)$ é um reticulado
    \end{itemize}
  \end{itemize}
\end{frame}

\begin{frame}
  \frametitle{Exemplo}
  \framesubtitle{Reticulados}
  \centering
  \begin{tikzpicture}
    \node [small,          label=above:{\tiny f}] (f) {};
    \node [small, below    = of j, label=right:{\tiny e}] (e) {};
    \node [small, below    = of e, label=right:{\tiny c}] (c) {};
    \node [small, below left = of f, label=left:{\tiny d}] (d) {};
    \node [small, below    = of d, label=left:{\tiny b}] (b) {};
    \node [small, below left = of c, label=below:{\tiny a}] (a) {};
    \draw (a) edge (b) (a) edge (c) (d) edge (b) (d) edge (f)
    (e) edge (c) (e) edge (f) (b) edge (e) (c) edge (d);
  \end{tikzpicture} \qquad
  \begin{tikzpicture}
    \node [small,           label=above:{\tiny h}] (h) {};
    \node [small, below left  = of h, label=left:{\tiny e}] (e) {};
    \node [small, below     = of h, label=right:{\tiny f}] (f) {};
    \node [small, below right = of h, label=right:{\tiny g}] (g) {};
    \node [small, below     = of e, label=left:{\tiny b}] (b) {};
    \node [small, below     = of f, label=right:{\tiny c}] (c) {};
    \node [small, below     = of g, label=right:{\tiny d}] (d) {};
    \node [small, below     = of c, label=below:{\tiny a}] (a) {};
    \draw (g) edge (h) (e) edge (h) (f) edge (h) (e) edge (b) (f) edge (c)
    (g) edge (d) (b) edge (a) (c) edge (a) (d) edge (a);
  \end{tikzpicture}
  \begin{itemize}
    \item Considere os \emph{posets} acima. Estes são reticulados?
    \begin{itemize}[itemsep=0pt]
      \item<2-> O primeiro \emph{poset} não é um reticulado, pois não existe
          $\inf(\{b, c\})$, enquanto o segundo \emph{poset} é
    \end{itemize}
  \end{itemize}
\end{frame}

\begin{frame}
  \frametitle{Reticulados limitados}
  \begin{itemize}
    \item Um reticulado que possui elementos máximo e mínimo, ou seja, $\top$ e
        $\bot$, é chamado de \textbf{limitado}
    \item Respectivamente, são os elementos identidade para as operações de
        encontro e junção
    \item<2-> Portanto, dado um reticulado $(S, \preccurlyeq)$, e $\forall s
        \in S$,
    \begin{itemize}[itemsep=0pt]
      \item $\bot \preccurlyeq s \preccurlyeq \top$
      \item $s \vee \bot = s, \quad s \wedge \bot = \bot$
      \item $s \wedge \top = s, \quad s \vee \top = \top$
    \end{itemize}
  \end{itemize}
\end{frame}

\begin{frame}
  \frametitle{Reticulados completos}
  \begin{itemize}
    \item Um reticulado onde todos os seus subconjuntos têm supremo e ínfimo é
        chamado de \textbf{completo}
    \item Formalmente, para um reticulado $(S, \preccurlyeq),\: \forall T \in
        \mathcal{P}(S)$, $\wedge T$ e $\vee T$ existem
    \item De maneira equivalente, um reticulado é completo se é um
        \emph{join-semilattice} completo e um \emph{meet-semilattice} completo
          ao mesmo tempo
    \item Todo reticulado completo é também limitado
  \end{itemize}
\end{frame}

\begin{frame}
  \frametitle{Sub-reticulados}
  \begin{itemize}
    \item Dado um reticulado $(S, \preccurlyeq)$, um \textbf{sub-reticulado},
        ou \emph{sublattice}, é um subconjunto não-vazio finito de $S$ com as
          operações de junção e encontro herdadas de $\preccurlyeq$
    \item Exemplo: tome o reticulado abaixo
    \begin{figure}
      \begin{tikzpicture}
        \node [small, label=above:{\tiny $\top$}] (xyz) at (0, 0) {};
        \node [small, below left=of xyz, label=left:{\tiny e}] (xy) {};
        \node [small, below right=of xyz, label=right:{\tiny f}] (yz) {};
        \node [small, below=of xyz, label=below:{\tiny c}] (xz) {};
        \node [small, below=of xy, label=left:{\tiny a}] (x) {};
        \node [small, below=of yz, label=right:{\tiny b}] (z) {};
        \node [below=of xz] (y) {};
        \node [small, below=of y, label=below:{\tiny $\bot$}] (0) {};
        \draw (0)  edge (x)  (0)  edge (z)  (x)  edge (xy)  (x)  edge (xz)
              (z)  edge (xz) (z)  edge (yz) (xy) edge (xyz) (xz) edge (xyz)
              (yz) edge (xyz);
      \end{tikzpicture}
    \end{figure}
  \end{itemize}
\end{frame}

\begin{frame}
  \frametitle{Sub-reticulados}
  \begin{itemize}
    \item<1-> O subconjunto parcialmente ordenado abaixo é um sub-reticulado do
        reticulado apresentado anteriormente?
    \begin{figure}
      \begin{tikzpicture}
        \node [small, label=above:{\tiny $\top$}] (xyz) at (0, 0) {};
        \node [small, below left=of xyz, label=left:{\tiny e}] (xy) {};
        \node [small, below right=of xyz, label=right:{\tiny f}] (yz) {};
        \node [below=of xyz] (xz) {};
        \node [small, below=of xy, label=left:{\tiny a}] (x) {};
        \node [small, below=of yz, label=right:{\tiny b}] (z) {};
        \node [below=of xz] (y) {};
        \node [small, below=of y, label=below:{\tiny $\bot$}]
          (0) at (0, -2.12) {};
        \draw (0)  edge (x)  (0)  edge (z)   (x)  edge (xy)
              (z)  edge (yz) (xy) edge (xyz) (yz) edge (xyz);
      \end{tikzpicture}
    \end{figure}
    \item<2-> Não, pois $a \vee b = c$, que não está presente
    \begin{itemize}
        \item Entretanto, é um reticulado por si só
    \end{itemize}
  \end{itemize}
\end{frame}

\begin{frame}
  \frametitle{Sub-reticulados}
  \begin{itemize}
    \item Os subconjuntos parcialmente ordenados abaixo são sub-reticulados do
        reticulado apresentado anteriormente?
    \begin{figure}
      \begin{tikzpicture}
        \node [small, label=above:{\tiny c}] (xz) at (0, 0) {};
        \node [small, below left=of xz, label=left:{\tiny a}] (x) {};
        \node [small, below right=of xz, label=right:{\tiny b}] (z) {};
        \node [small, below=of xz, label=below:{\tiny $\bot$}]
          (0) at (0, -1.12) {};
        \draw (0) edge (x) (0) edge (z) (x) edge (xz) (z) edge (xz);
      \end{tikzpicture}
      \begin{tikzpicture}
        \node [small, label=above:{\tiny $\top$}] (xyz) at (0, 0) {};
        \node [small, below left=of xyz, label=left:{\tiny e}] (xy) {};
        \node [small, below right=of xyz, label=right:{\tiny f}] (yz) {};
        \node [below=of xyz] (xz) {};
        \node [small, below=of xy, label=left:{\tiny a}] (x) {};
        \node [small, below=of yz, label=right:{\tiny b}] (z) {};
        \node [below=of xz] (y) {};
        \draw (x) edge (xy) (z) edge (yz) (xy) edge (xyz) (yz) edge (xyz);
      \end{tikzpicture}
    \end{figure}
    \item Respectivamente, sim, e não, pois $a \wedge b$ e $a \vee b$ não estão
        presentes
  \end{itemize}
\end{frame}

\begin{frame}
  \frametitle{Isomorfismo entre reticulados}
  \begin{itemize}
    \item<1-> Uma função bijetora que mapeia elementos de um reticulado para
        outro pode ser chamada de isomorfismo
    \item<2-> Formalmente, tome dois reticulados $(S_{1},
        \textcolor{green}{\preccurlyeq}), (S_{2},
          \textcolor{red}{\preccurlyeq})$, uma função $f : S_{1} \rightarrow
          S_{2}$, e elementos quaisquer $a, b \in S_{1}$
    \item<3-> Para que exista um isomorfismo, $f$ deve preservar as operações
        de junção e encontro
    \begin{itemize}[itemsep=0pt]
      \item<4-> $f(a \mathbin{\textcolor{green}{\vee}} b) = f(a)
          \mathbin{\textcolor{red}{\vee}} f(b)$ (isomorfismo de junção)
      \item<5-> $f(a \mathbin{\textcolor{green}{\wedge}} b) = f(a)
          \mathbin{\textcolor{red}{\wedge}} f(b)$ (isomorfismo de encontro)
    \end{itemize}
  \end{itemize}
\end{frame}

\begin{frame}
  \frametitle{Exemplo}
  \framesubtitle{Isomorfismo entre reticulados}
  \begin{figure}
    \begin{tikzpicture}
      \node [small, label=above:{\tiny 30}] (xyz) at (0, 0) {};
      \node [small, below left=of xyz, label=left:{\tiny 6}] (xy) {};
      \node [small, below=of xyz, label=right:{\tiny 10}] (xz) {};
      \node [small, below right=of xyz, label=right:{\tiny 15}] (yz) {};
      \node [small, below=of xy, label=left:{\tiny 2}] (x) {};
      \node [small, below=of xz, label=right:{\tiny 3}] (y) {};
      \node [small, below=of yz, label=right:{\tiny 5}] (z) {};
      \node [small, below=of y, label=below:{\tiny 1}] (0) {};
      \draw (0) edge (x)  (0)  edge (y)   (0)  edge (z)   (x)  edge (xy)
            (x) edge (xz) (y)  edge (xy)  (y)  edge (yz)  (z)  edge (xz)
            (z) edge (yz) (xy) edge (xyz) (xz) edge (xyz) (yz) edge (xyz);
    \end{tikzpicture} \quad
    \begin{tikzpicture}
      \node [small, label=above:{\tiny $\{x, y, z\}$}] (xyz) at (0, 0) {};
      \node [small, below left=of xyz, label=left:{\tiny $\{x, y\}$}] (xy) {};
      \node [small, below right=of xyz, label=right:{\tiny $\{y, z\}$}] (yz) {};
      \node [small, below=of xyz, label=right:{\tiny $\{x, z\}$}] (xz) {};
      \node [small, below=of xy, label=left:{\tiny $\{x\}$}] (x) {};
      \node [small, below=of xz, label=right:{\tiny $\{y\}$}] (y) {};
      \node [small, below=of yz, label=right:{\tiny $\{z\}$}] (z) {};
      \node [small, below=of y, label=below:{\tiny $\emptyset$}] (0) {};
      \draw (0) edge (x)  (0)  edge (y)   (0)  edge (z)   (x)  edge (xy)
            (x) edge (xz) (y)  edge (xy)  (y)  edge (yz)  (z)  edge (xz)
            (z) edge (yz) (xy) edge (xyz) (xz) edge (xyz) (yz) edge (xyz);
    \end{tikzpicture}
  \end{figure}
  \begin{itemize}
    \item Os reticulados acima são isomórficos, pois, por exemplo, $f(2 \vee 5)
        = f(2) \vee f(5) \Rightarrow \{x, z\} =  \{x\} \vee \{z\}$
  \end{itemize}
\end{frame}

\begin{frame}
  \frametitle{Reticulados como estruturas algébricas}
  \begin{itemize}
    \item<1-> Operações binárias através de supremos e ínfimos
    \item<1-> Identidades através da possível presença de elementos máximo e
        mínimo
    \item<2-> Uma álgebra é uma tupla composta de um conjunto e operações de
        aridade finita e relações
    \begin{itemize}[itemsep=0pt]
      \item Reticulados admitem uma descrição algébrica
      \item Notação equivalente: $(S, \preccurlyeq) \Leftrightarrow (S, \vee,
          \wedge)$
    \end{itemize}
  \end{itemize}
\end{frame}

\begin{frame}
  \frametitle{Reticulados como estruturas algébricas}
  \begin{itemize}
    \item<1-> Consolidando as definições mostradas anteriormente
    \item<1-> Para um reticulado $(S, \preccurlyeq)$, existe um reticulado $(S,
        \vee, \wedge)$ de tal modo que, $\forall s_1, s_2 \in S$
    \begin{itemize}[itemsep=0pt]
      \item<2-> $\sup(\{s_1, s_2\}) \Leftrightarrow s_1 \vee s_2$
      \item<3-> $\inf(\{s_1, s_2\}) \Leftrightarrow s_1 \wedge s_2$
      \item<4-> $s_1 \preccurlyeq s_2 \Leftrightarrow \quad s_1 \vee s_2 = s_2,
          s_1 \wedge s_2 = s_1$
    \end{itemize}
    \item<5-> No caso de um reticulado limitado, existe um reticulado $(S, \vee,
        \wedge, \bot, \top)$ de tal modo que, adicionalmente,
    \begin{itemize}[itemsep=0pt]
      \item<6-> $\bot \preccurlyeq s_1 \preccurlyeq \top$
      \item<7-> $s_1 \vee \bot = s_1, \quad s_1 \wedge \bot = \bot$
      \item<8-> $s_1 \wedge \top = s_1, \quad s_1 \vee \top = \top$
    \end{itemize}
  \end{itemize}
\end{frame}

\begin{frame}
  \frametitle{Identidades axiomáticas para um reticulado}
  \begin{itemize}
    \item<1-> Para um reticulado qualquer $(S, \vee, \wedge)$, e $\forall a, b,
        c \in S$
    \begin{itemize}[itemsep=0pt]
      \item<2-> $a \vee b = b \vee a,\; a \wedge b = b \wedge a$
          (comutatividade)
      \item<3-> $a \vee (b \vee c) = (a \vee b) \vee c,\; a \wedge (b \wedge c)
          = (a \wedge b) \wedge c$ (associatividade)
      \item<4-> $a \vee (a \wedge b) = a,\; a \wedge (a \vee b) = a$ (absorção)
      \item<5-> $a \vee a = a,\; a \wedge a = a$ (idempotência, derivada da
          absorção)
    \end{itemize}
    \item<6-> Para um reticulado limitado qualquer $(S, \vee, \wedge, \bot,
        \top)$
    \begin{itemize}
      \item<7-> $a \vee \bot = a,\; a \wedge \top = a$ (identidade)
    \end{itemize}
  \end{itemize}
\end{frame}

\begin{frame}
  \frametitle{Reticulados complementados}
  \begin{itemize}
    \item Um reticulado qualquer $(S, \vee, \wedge, \bot, \top)$ é
        \textbf{complementado} se, $\forall a, b \in S$, $a \vee b = \top$ e $a
          \wedge b = \bot$
    \item Então, $a$ é complemento de $b$ e vice-versa
    \begin{itemize}
      \item Um complemento de $a$ pode ser denotado por $a^{\bot}$
    \end{itemize}
    \item Complementos não necessariamente são únicos
    \begin{itemize}
      \item Dois elementos estão relacionados se têm um complemento em comum
    \end{itemize}
  \end{itemize}
\end{frame}

\begin{frame}
  \frametitle{Reticulados distributivos}
  \begin{itemize}
    \item<1-> Um reticulado qualquer $(S, \vee, \wedge, \bot, \top)$ é
        \textbf{distributivo} se, $\forall a, b, c \in S$,
    \begin{itemize}[itemsep=0pt]
      \item<2-> $a \vee (b \wedge c) = (a \vee b) \wedge (a \vee c)$
          (distribuição de $\vee$ sobre $\wedge$)
      \item<3-> $a \wedge (b \vee c) = (a \wedge b) \vee (a \wedge c)$
          (distribuição de $\wedge$ sobre $\vee$)
    \end{itemize}
    \item Todo elemento de um reticulado distributivo terá até um complemento
    \item Todo reticulado distribuído é isomórfico a um reticulado
        $(\mathcal{P}(S), \cup,\, \cap,\, \emptyset, S)$, para qualquer $S$
  \end{itemize}
\end{frame}

\begin{frame}
  \frametitle{Exemplo}
  \framesubtitle{Reticulados distributivos}
 \begin{figure}
    \begin{tikzpicture}
      \node [small, below=of xyz, label=above:{\tiny $\top$}] (xz) {};
      \node [small, below=of xy, label=left:{\tiny x}] (x) {};
      \node [small, below=of xz, label=right:{\tiny y}] (y) {};
      \node [small, below=of yz, label=right:{\tiny z}] (z) {};
      \node [small, below=of y, label=below:{\tiny $\bot$}] (0) {};
      \draw (0) edge (x)  (0) edge (y)  (0)  edge (z)
            (x) edge (xz) (y) edge (xz) (z)  edge (xz);
    \end{tikzpicture} \qquad
    \begin{tikzpicture}
      \node [small, label=above:{\tiny $\top$}] (xyz) at (0, 0) {};
      \node [small, below left=of xyz, label=left:{\tiny x}] (xy) {};
      \node [small, below=of yz, label=right:{\tiny y}] (z) {};
      \node [small, below=of xy, label=left:{\tiny z}] (x) {};
      \node [small, below=of y, label=below:{\tiny $\bot$}] (0) {};
      \draw (0)  edge (x)  (0)  edge (z) (x)  edge (xy) (z)  edge (xyz)
            (xy) edge (xyz);
    \end{tikzpicture}
  \end{figure}
  \begin{itemize}
    \item<1-> Um reticulado é distributivo se e somente se não ter um
        sub-reticulado isomórfico a $M_{3}$ ou $N_{5}$
    \begin{itemize}[itemsep=0pt]
      \item<2-> $M_{3}: x \wedge (y \vee z) = x \wedge 1 \neq 0 \vee 0 = (x
          \wedge y) \vee (x \wedge z)$
      \item<3-> $N_{5}: x \wedge (y \vee z) = x \wedge 1 \neq 0 \vee z = (x
          \wedge y) \vee (x \wedge z)$
    \end{itemize}
  \end{itemize}
\end{frame}

\begin{frame}
  \frametitle{Material de estudo}
  \bibliographystyle{apalike}
  \bibliography{ref}
  \begin{itemize}[itemsep=0pt]
    \nocite{Kolman:book:1999}
    \item Kolman: leitura das páginas 207-215 e resolução dos exercícios 1-7,
        11, 13-16, 20, na página 216
    \nocite{Rosen:book:2011}
    \item Rosen: leitura das páginas 626-627 e resolução dos exercícios 43-44,
        46-48, 50-52, na página 632
  \end{itemize}
\end{frame}

\end{document}
