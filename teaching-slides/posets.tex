\documentclass[12pt]{beamer}

\usepackage[brazil]{babel}
\usepackage[T1]{fontenc}
\usepackage[utf8]{inputenc}
\usepackage{enumitem, tikz}

\setitemize{
  itemsep=1em,
  label=\usebeamerfont*{itemize item}
    \usebeamercolor[fg]{itemize item}
    \usebeamertemplate{itemize item}
}

\usetikzlibrary{positioning, decorations.markings}
\tikzset{small/.style={draw,fill,circle,inner sep=1pt,outer sep=0pt}}

\setbeamertemplate{footline}[frame number]{}
\setbeamertemplate{navigation symbols}{}

\title{Introdução à teoria da ordem}
\author{Gustavo Zambonin}
\institute{
  \includegraphics[scale=0.15]{ufsc}                    \\ \vspace{-4mm}
  Universidade Federal de Santa Catarina                \\
  Departamento de Informática e Estatística             \\
  INE5601 --- Fundamentos Matemáticos da Informática    \\ \vspace{2mm}
  \texttt{gustavo.zambonin@posgrad.ufsc.br}
}
\date{}

\begin{document}

\begin{frame}[plain,noframenumbering]
  \titlepage
\end{frame}

\begin{frame}
  \frametitle{Contexto}
  \begin{itemize}
    \item Relação: subconjunto do produto cartesiano de dois (ou mais)
        conjuntos quaisquer
    \begin{itemize}
      \item $\equiv_n$, a congruência módulo $n$ em $\mathbb{Z}$, ou seja,
          $x \equiv y \pmod{n}$
    \end{itemize}
    \item Propriedades de relações: reflexividade, irreflexividade, simetria,
        assimetria, antissimetria, transitividade
    \begin{itemize}
      \item $\equiv_n$ é uma relação de equivalência
    \end{itemize}
    \item É intuitivo pensar que relações são utilizadas para construir
        ordenamentos
  \end{itemize}
\end{frame}

\begin{frame}
  \frametitle{Exemplos práticos}
  \begin{itemize}
    \item Organização de verbetes em um dicionário
    \begin{itemize}
      \item Letras no alfabeto e tamanho do verbete
    \end{itemize}
    \item Descrição de um grafo curricular
    \begin{itemize}
      \item Matérias como pré-requisitos
    \end{itemize}
    \item Parentesco entre pessoas, escalonamento temporal de tarefas etc.
    \item Ideia geral: comparação de (alguns) elementos entre conjuntos
        quaisquer
  \end{itemize}
\end{frame}

\begin{frame}
  \frametitle{Propriedades de relações}
  \begin{itemize}
    \item Considere um conjunto $S$ e uma relação $R$, construída a partir de
        $S \times S$
    \item $R$ é considerada uma relação \emph{parcial} se é reflexiva,
        antissimétrica e transitiva
    \item Ou seja, $\forall a, b, c \in S$,
    \begin{itemize}[itemsep=0pt]
      \item $(a, a) \in R$ (reflexividade)
      \item $(a, b), (b, a) \in R \Rightarrow a = b$ (antissimetria)
      \item $(a, b), (b, c) \in R \Rightarrow (a, c) \in R$ (transitividade)
    \end{itemize}
  \end{itemize}
\end{frame}

\begin{frame}
  \frametitle{Conjuntos parcialmente ordenados}
  \begin{itemize}
    \item Quando parcial, $R$ é geralmente denotada com o símbolo da relação,
        ou $\leq$, $\preccurlyeq$
    \item O par ordenado $(S, \preccurlyeq)$ é então chamado de conjunto
        parcialmente ordenado, ou \textbf{poset} (\emph{partially ordered set})
    \item $a, b$ são \textbf{comparáveis} se $a \preccurlyeq b$ ou $b
        \preccurlyeq a$, e incomparáveis caso contrário
    \begin{itemize}
      \item se $a \neq b$, a notação $a \prec b$ é utilizada
    \end{itemize}
  \end{itemize}
\end{frame}

\begin{frame}
  \frametitle{Exemplo}
  \framesubtitle{Conjuntos parcialmente ordenados}
  \begin{itemize}
    \item Considere o conjunto dos inteiros $\mathbb{Z}$ e a relação de ``menor
        ou igual'' $\leq$. $(\mathbb{Z}, \leq)$ é um \emph{poset}?
    \item Tome inteiros quaisquer $a, b, c \in \mathbb{Z}$
    \begin{itemize}[itemsep=0pt]
      \item $a \leq a$ para todo inteiro, portanto $\leq$ é reflexiva
      \item $a \leq b, b \leq a \Rightarrow a = b$, portanto $\leq$ é
          antissimétrica
      \item $a \leq b, b \leq c \Rightarrow a \leq c$, portanto $\leq$ é
          transitiva
    \end{itemize}
    \item Logo, $(\mathbb{Z}, \leq)$ é um \emph{poset}
  \end{itemize}
\end{frame}

\begin{frame}
  \frametitle{Exemplo}
  \framesubtitle{Conjuntos parcialmente ordenados}
  \begin{itemize}
    \item Considere um conjunto $T$ e seu conjunto potência, $\mathcal{P}(T)$,
        e a relação de ``subconjunto'' $\subseteq$. $(\mathcal{P}(T),
          \subseteq)$ é um \emph{poset}?
    \item Tome conjuntos quaisquer $t_1, t_2, t_3 \in \mathcal{P}(T)$
    \begin{itemize}[itemsep=0pt]
      \item $t_1 \subseteq t_1$ para todo conjunto, portanto $\subseteq$ é
          reflexiva
      \item $t_1 \subseteq t_2, t_2 \subseteq t_1 \Rightarrow t_1 = t_2$,
          portanto $\subseteq$ é antissimétrica
      \item $t_1 \subseteq t_2, t_2 \subseteq t_3 \Rightarrow t_1 \subseteq
          t_3$, portanto $\subseteq$ é transitiva
    \end{itemize}
    \item Logo, $(\mathcal{P}(T), \subseteq)$ é um \emph{poset}
  \end{itemize}
\end{frame}

\begin{frame}
  \frametitle{Exemplo}
  \framesubtitle{Conjuntos parcialmente ordenados}
  \begin{itemize}
    \item Considere o conjunto de seres humanos $H$, e a relação de ``é mais
        velho que'' $\preccurlyeq$. $(H, \preccurlyeq)$ é um \emph{poset}?
    \item Tome pessoas quaisquer $h_1, h_2, h_3 \in H$
    \begin{itemize}[itemsep=0pt]
      \item $h_1 \preccurlyeq h_2 \Rightarrow h_2 \not\preccurlyeq h_1$
          (antissimetria: se $h_1$ é mais velha que $h_2$, o oposto não pode
            ser verdade)
      \item $h_1 \preccurlyeq h_2, h_2 \preccurlyeq h_3 \Rightarrow h_1
          \preccurlyeq h_3$ (transitividade: se $h_1$ é mais velha que $h_2$, e
            $h_2$ é mais velha que $h_3$, então $h_1$ é mais velha que $h_3$)
      \item $h_1 \not\preccurlyeq h_1$ (reflexividade: uma pessoa não pode ser
          mais velha do que si mesma)
    \end{itemize}
    \item Portanto, $(H, \preccurlyeq)$ não é um \emph{poset}
  \end{itemize}
\end{frame}

\begin{frame}
  \frametitle{Conjuntos parcialmente ordenados estritos}
  \begin{itemize}
    \item Em alguns contextos, a definição de \emph{poset} apresentada é
        chamada de ``conjunto parcialmente ordenado não-estrito''
    \item O par ordenado $(S, R)$ é um conjunto parcialmente ordenado estrito
        se, $\forall a, b, c \in S$,
    \begin{itemize}[itemsep=0pt]
      \item $(a, a) \not\in R$ (irreflexividade)
      \item $(a, b), (b, c) \in R \Rightarrow (a, c) \in R$ (transitividade)
    \end{itemize}
    \item Para transformar um \emph{poset} em um \emph{poset} estrito, remova
      todos os pares $(a, a) \in R$ \item A notação para relações parciais
        estritas arbitrárias é similar, utilizando os símbolos $<, \prec$
  \end{itemize}
\end{frame}

\begin{frame}
  \frametitle{Exemplo}
  \framesubtitle{Conjuntos parcialmente ordenados estritos}
  \begin{itemize}
    \item Considere o conjunto de matérias em um currículo $C$, e a relação de
        ``é pré-requisito de'' $\prec$. $(C, \prec)$ é um \emph{poset} estrito?
    \item Tome matérias quaisquer $c_1, c_2, c_3 \in C$
    \begin{itemize}[itemsep=0pt]
      \item $c_1 \not\prec c_1$ (irreflexividade: $c_1$ não pode ser
          pré-requisito dela mesma)
      \item $c_1 \prec c_2, c_2 \prec c_3 \Rightarrow c_1 \prec c_3$
          (transitividade: se $c_1$ é pré-requisito de $c_2$, e $c_2$ é
            pré-requisito de $c_3$, então $c_1$ é pré-requisito de $c_3$)
    \end{itemize}
    \item Portanto, $(C, \prec)$ é um \emph{poset} estrito
  \end{itemize}
\end{frame}

\begin{frame}
  \frametitle{Dualidade de \emph{posets}}
  \begin{itemize}
    \item Para qualquer relação binária $R$, uma relação inversa $R^{-1}$ pode
        ser construída trocando a ordem dos elementos de todos os pares
          ordenados em $R$
    \begin{itemize}
      \item As propriedades usuais de relações mantêm-se em $R^{-1}$
    \end{itemize}
    \item Formalmente, para dois conjuntos quaisquer $S, T$, $R^{-1} = \{(t, s)
        \in T \times S \mid (s, t) \in R\}$
    \item No caso de relações parciais, os símbolos $\succcurlyeq, \geq, \succ,
        >$ podem ser utilizados para denotar a inversa, neste caso chamada de
          \textbf{dual}
  \end{itemize}
\end{frame}

\begin{frame}
  \frametitle{Exemplo}
  \framesubtitle{Conjuntos parcialmente ordenados}
  \begin{itemize}
    \item Considere o conjunto dos inteiros $\mathbb{Z}$ e a relação de ``maior
        ou igual'' $\geq$. $(\mathbb{Z}, \geq)$ é um \emph{poset}
    \item Considere um conjunto $T$ e seu conjunto potência, $\mathcal{P}(T)$,
        e a relação de ``superconjunto'' $\supseteq$.  $(\mathcal{P}(T),
          \supseteq)$ é um \emph{poset}
    \item Considere o conjunto de matérias em um currículo $C$, e a relação de
        ``é pós-requisito de'' $\succ$. $(C, \succ)$ é um \emph{poset} estrito
  \end{itemize}
\end{frame}

\begin{frame}
  \frametitle{Conjuntos totalmente ordenados}
  \begin{itemize}
    \item Quando todos os elementos de uma relação parcial são comparáveis,
        esta é chamada de \textbf{ordem total}
    \item O par ordenado $(S, R)$ é chamado de conjunto totalmente ordenado, ou
        \textbf{toset} (\emph{totally ordered set}) se, $\forall a, b \in S$,
    \begin{itemize}[itemsep=0pt]
      \item $R$ é uma relação parcial
      \item Ou $(a, b) \in R$, ou $(b, a) \in R$, ou $a = b$ (tricotomia)
    \end{itemize}
    \item São um caso particular de \emph{posets}, e portanto, a dualidade
        mantém-se
    \item Também podem ser chamados de \textbf{cadeias} e ordens lineares, e
        seus duais, de anticadeias
  \end{itemize}
\end{frame}

\begin{frame}
  \frametitle{Exemplo}
  \framesubtitle{Conjuntos totalmente ordenados}
  \begin{itemize}
    \item Considere o conjunto dos inteiros positivos $\mathbb{Z}^{+}$ e a
        relação de divisibilidade $\mid$. $(\mathbb{Z}^{+}, \mid)$ é um
          \emph{poset}? Se sim, é também um \emph{toset}?
    \item Para quaisquer inteiros $a, b, c \in \mathbb{Z}^{+}$, e quaisquer
        dois primos $p_1, p_2 \in \mathbb{Z}^{+}$
    \begin{itemize}[itemsep=0pt]
      \item $a \mid a$ (reflexividade)
      \item $a \mid b, b \mid a \Rightarrow a = b$ (antissimetria)
      \item $a \mid b, b \mid c \Rightarrow a \mid c$ (transitividade)
      \item $p_1 \nmid p_2, p_2 \nmid p_1$ (existem elementos incomparáveis)
    \end{itemize}
    \item Então, $(\mathbb{Z}^{+}, \mid)$ é um \emph{poset}, mas não um
        \emph{toset}
  \end{itemize}
\end{frame}

\begin{frame}
  \frametitle{Exemplo}
  \framesubtitle{Conjuntos totalmente ordenados}
  \begin{itemize}
    \item O \emph{poset} $(\mathbb{Z}, \leq)$ é também um \emph{toset}, pois
        quaisquer inteiros podem ser comparados com $\leq$
    \item O \emph{poset} $(\mathcal{P}(T), \subseteq)$ não é um \emph{toset}
    \begin{itemize}
      \item Considere $T = \{x, y\}, \{x\}, \{y\} \in \mathcal{P}(T)$ e veja
          que $\{x\} \not\subseteq \{y\}, \{y\} \not\subseteq \{x\}$
    \end{itemize}
    \item O produto cartesiano de qualquer número de \emph{tosets}, utilizando
        ordenação lexicográfica, é um \emph{toset}
    \begin{itemize}[itemsep=0pt]
      \item Dados dois \emph{tosets} $(P, \preccurlyeq_1), (Q,
          \preccurlyeq_2)$, e elementos quaisquer $p_1, p_2 \in P, q_1, q_2 \in
            Q$
      \item Para $(P_1 \times P_2, \preccurlyeq)$, a relação $(p_1, q_1)
          \preccurlyeq (p_2, q_2) \Rightarrow p_1 \prec_1 p_2 \text{ ou } (p_1
            = p_2 \text{ e } q_1 \preccurlyeq_2 q_2)$
    \end{itemize}
  \end{itemize}
\end{frame}

\begin{frame}
  \frametitle{Representação gráfica de \emph{posets}}
  \begin{itemize}
    \item \emph{Posets} podem ser caracterizados de maneira amigável,
        considerando os elementos de $S$ como ``pontos em um plano''
    \item As conexões entre elementos são análogas à descrição das relações, e
        utilizam-se ``vetores'' para simbolizá-las
    \item De maneira formal, um \emph{poset} pode ser representado como um
        grafo dirigido, cujos vértices são os elementos de $S$, e arestas os
          elementos de $\preccurlyeq$
  \end{itemize}
\end{frame}

\begin{frame}
  \frametitle{Diagrama de Hasse}
  \begin{itemize}
    \item Deseja-se desenhar o \emph{poset} com o menor número de conexões
        possível (através da redução transitiva do grafo correspondente)
    \item Esta técnica gera um esquema chamado de \textbf{diagrama de Hasse},
        que representa o \emph{poset} de maneira sucinta
    \item Entretanto, existem várias formas de desenhá-los, tornando-se difícil
        criar diagramas ordenados
    \item O diagrama de Hasse de \emph{tosets} dá origem ao nome de ``cadeia''
  \end{itemize}
\end{frame}

\begin{frame}
  \frametitle{Exemplo}
  \framesubtitle{Diagrama de Hasse}
  \centering
  \resizebox{6cm}{!}{
    \begin{tikzpicture}
      \node [small, label=above:{\tiny $\{x, y, z\}$}] (xyz) at (0, 0) {};
      \node [small, below left=of xyz, label=left:{\tiny $\{x, y\}$}] (xy) {};
      \node [small, below right=of xyz, label=right:{\tiny $\{y, z\}$}] (yz) {};
      \node [small, below=of xyz, label=right:{\tiny $\{x, z\}$}] (xz) {};
      \node [small, below=of xy, label=left:{\tiny $\{x\}$}] (x) {};
      \node [small, below=of xz, label=right:{\tiny $\{y\}$}] (y) {};
      \node [small, below=of yz, label=right:{\tiny $\{z\}$}] (z) {};
      \node [small, below=of y, label=below:{\tiny $\emptyset$}] (0) {};
      \draw[->] (0)   edge                                       (x)
                (0)   edge                                       (y)
                (0)   edge                                       (z)
                (x)   edge                                       (xy)
                (x)   edge                                       (xz)
                (y)   edge                                       (xy)
                (y)   edge                                       (yz)
                (z)   edge                                       (xz)
                (z)   edge                                       (yz)
                (xy)  edge                                       (xyz)
                (xz)  edge                                       (xyz)
                (yz)  edge                                       (xyz)
                (0)   edge                                       (xy)
                (0)   edge [bend right]                          (yz)
                (0)   edge [bend left]                           (xz)
                (0)   edge [bend left]                           (xyz)
                (x)   edge                                       (xyz)
                (y)   edge [bend left]                           (xyz)
                (z)   edge [bend left]                           (xyz)
                (0)   edge [loop, in=225, out=180, looseness=20] (0)
                (x)   edge [loop, in=270, out=225, looseness=20] (x)
                (y)   edge [loop, in=225, out=180, looseness=20] (y)
                (z)   edge [loop, in=270, out=315, looseness=20] (z)
                (xy)  edge [loop, in=90,  out=135, looseness=20] (xy)
                (yz)  edge [loop, in=90,  out=45,  looseness=20] (yz)
                (xz)  edge [loop, in=135, out=90,  looseness=20] (xz)
                (xyz) edge [loop, in=180, out=135, looseness=20] (xyz);
    \end{tikzpicture}
  }
  \begin{itemize}
    \item $(\mathcal{P}(\{x, y, z\}), \subseteq)$ com todas as arestas
        relativas às relações entre elementos
  \end{itemize}
\end{frame}

\begin{frame}
  \frametitle{Exemplo}
  \framesubtitle{Diagrama de Hasse}
  \centering
  \resizebox{6cm}{!}{
    \begin{tikzpicture}
      \node [small, label=above:{\tiny $\{x, y, z\}$}] (xyz) at (0, 0) {};
      \node [small, below left=of xyz, label=left:{\tiny $\{x, y\}$}] (xy) {};
      \node [small, below right=of xyz, label=right:{\tiny $\{y, z\}$}] (yz) {};
      \node [small, below=of xyz, label=right:{\tiny $\{x, z\}$}] (xz) {};
      \node [small, below=of xy, label=left:{\tiny $\{x\}$}] (x) {};
      \node [small, below=of xz, label=right:{\tiny $\{y\}$}] (y) {};
      \node [small, below=of yz, label=right:{\tiny $\{z\}$}] (z) {};
      \node [small, below=of y, label=below:{\tiny $\emptyset$}] (0) {};
      \draw[->] (0)   edge                                       (x)
                (0)   edge                                       (y)
                (0)   edge                                       (z)
                (x)   edge                                       (xy)
                (x)   edge                                       (xz)
                (y)   edge                                       (xy)
                (y)   edge                                       (yz)
                (z)   edge                                       (xz)
                (z)   edge                                       (yz)
                (xy)  edge                                       (xyz)
                (xz)  edge                                       (xyz)
                (yz)  edge                                       (xyz)
                (0)   edge [loop, in=225, out=180, looseness=20] (0)
                (x)   edge [loop, in=270, out=225, looseness=20] (x)
                (y)   edge [loop, in=225, out=180, looseness=20] (y)
                (z)   edge [loop, in=270, out=315, looseness=20] (z)
                (xy)  edge [loop, in=90,  out=135, looseness=20] (xy)
                (yz)  edge [loop, in=90,  out=45,  looseness=20] (yz)
                (xz)  edge [loop, in=135, out=90,  looseness=20] (xz)
                (xyz) edge [loop, in=180, out=135, looseness=20] (xyz);
    \end{tikzpicture}
  }
  \begin{itemize}
    \item As arestas relativas à transitividade da relação são omitidas
  \end{itemize}
\end{frame}

\begin{frame}
  \frametitle{Exemplo}
  \framesubtitle{Diagrama de Hasse}
  \centering
  \resizebox{6cm}{!}{
    \begin{tikzpicture}
      \node [small, label=above:{\tiny $\{x, y, z\}$}] (xyz) at (0, 0) {};
      \node [small, below left=of xyz, label=left:{\tiny $\{x, y\}$}] (xy) {};
      \node [small, below right=of xyz, label=right:{\tiny $\{y, z\}$}] (yz) {};
      \node [small, below=of xyz, label=right:{\tiny $\{x, z\}$}] (xz) {};
      \node [small, below=of xy, label=left:{\tiny $\{x\}$}] (x) {};
      \node [small, below=of xz, label=right:{\tiny $\{y\}$}] (y) {};
      \node [small, below=of yz, label=right:{\tiny $\{z\}$}] (z) {};
      \node [small, below=of y, label=below:{\tiny $\emptyset$}] (0) {};
      \draw[->] (0) edge (x)  (0)  edge (y)   (0)  edge (z)   (x)  edge (xy)
                (x) edge (xz) (y)  edge (xy)  (y)  edge (yz)  (z)  edge (xz)
                (z) edge (yz) (xy) edge (xyz) (xz) edge (xyz) (yz) edge (xyz);
    \end{tikzpicture}
  }
  \begin{itemize}
    \item As arestas relativas à reflexividade da relação também são omitidas
  \end{itemize}
\end{frame}

\begin{frame}
  \frametitle{Exemplo}
  \framesubtitle{Diagrama de Hasse}
  \centering
  \resizebox{6cm}{!}{
    \begin{tikzpicture}
      \node [small, label=above:{\tiny $\{x, y, z\}$}] (xyz) at (0, 0) {};
      \node [small, below left=of xyz, label=left:{\tiny $\{x, y\}$}] (xy) {};
      \node [small, below right=of xyz, label=right:{\tiny $\{y, z\}$}] (yz) {};
      \node [small, below=of xyz, label=right:{\tiny $\{x, z\}$}] (xz) {};
      \node [small, below=of xy, label=left:{\tiny $\{x\}$}] (x) {};
      \node [small, below=of xz, label=right:{\tiny $\{y\}$}] (y) {};
      \node [small, below=of yz, label=right:{\tiny $\{z\}$}] (z) {};
      \node [small, below=of y, label=below:{\tiny $\emptyset$}] (0) {};
      \draw (0) edge (x)  (0)  edge (y)   (0)  edge (z)   (x)  edge (xy)
            (x) edge (xz) (y)  edge (xy)  (y)  edge (yz)  (z)  edge (xz)
            (z) edge (yz) (xy) edge (xyz) (xz) edge (xyz) (yz) edge (xyz);
    \end{tikzpicture}
  }
  \begin{itemize}
    \item O direcionamento é omitido, visto que todas as arestas apontam ``para
        cima''
  \end{itemize}
\end{frame}

\begin{frame}
  \frametitle{Elementos extremos}
  \begin{itemize}
    \item É razoável conhecer os elementos que denotam as fronteiras de um
        \emph{poset} $(S, \preccurlyeq)$
    \item Considere um elemento qualquer $s' \in S$, e $\forall s \in S$
    \begin{itemize}[itemsep=0pt]
      \item $s'$ é \textbf{máximo} se é ``maior que'' todos os outros, ou seja,
          $s \prec s'$
      \item $s'$ é \textbf{mínimo} se é ``menor que'' todos os outros, ou seja,
          $s' \prec s$
      \item $s'$ é \textbf{maximal} se não existe outro ``maior que'' este, ou
          seja, $s' \preccurlyeq s$
      \item $s'$ é \textbf{minimal} se não existe outro ``menor que'' este, ou
          seja, $s \preccurlyeq s'$
    \end{itemize}
  \end{itemize}
\end{frame}

\begin{frame}
  \frametitle{Elementos extremos}
  \begin{itemize}
    \item Portanto, devem existir até um elemento máximo e um elemento mínimo
    \item Se este for o caso, estes serão o único elemento maximal e minimal, e
        serão denotados por $\top, \bot$ respectivamente
    \item À ausência destes, não existem restrições no número de elementos
        minimais ou maximais, e estes conjuntos serão denotados por $S_{\min},
          S_{\max}$ respectivamente
    \item Poderão existir elementos maximal e minimal únicos, mas não
        necessariamente serão o máximo e mínimo
  \end{itemize}
\end{frame}

\begin{frame}
  \frametitle{Exemplo}
  \framesubtitle{Elementos extremos}
  \begin{itemize}
    \item Considere o \emph{poset} $(\mathbb{Z}^{+}, \mid)$. É possível
        identificar elementos extremos?
    \begin{itemize}[itemsep=0pt]
      \item Como o \emph{poset} tende a $+\infty$, não existem elementos
          maximais e máximo
      \item $\bot = 1$, pois $1 \mid n, \forall n \in \mathbb{Z}^{+}$
    \end{itemize}
\item Considere o \emph{poset} $(\mathcal{P}(T), \subseteq)$. É possível
        identificar elementos extremos?
    \begin{itemize}[itemsep=0pt]
      \item Tome um subconjunto qualquer $t \in \mathcal{P}(T)$
      \item Observe que $\emptyset \subseteq t$, e portanto $\bot = \emptyset$
      \item Da mesma maneira, $t \subseteq T$, e portanto $\top = T$
    \end{itemize}
  \end{itemize}
\end{frame}

\begin{frame}
  \frametitle{Exemplo}
  \framesubtitle{Elementos extremos}
  \centering
  \resizebox{6cm}{!}{
    \begin{tikzpicture}
      \node [small, below left=of xyz, label=above:{\tiny $\{x, y\}$}] (xy) {};
      \node [small, below right=of xyz, label=above:{\tiny $\{y, z\}$}] (yz) {};
      \node [small, below=of xyz, label=above:{\tiny $\{x, z\}$}] (xz) {};
      \node [small, below=of xy, label=below:{\tiny $\{x\}$}] (x) {};
      \node [small, below=of xz, label=below:{\tiny $\{y\}$}] (y) {};
      \node [small, below=of yz, label=below:{\tiny $\{z\}$}] (z) {};
      \draw (x) edge (xy) (x) edge (xz) (y) edge (xy)
            (y) edge (yz) (z) edge (xz) (z) edge (yz);
    \end{tikzpicture}
  }
  \begin{itemize}
    \item Considere o diagrama de Hasse acima. É possível identificar elementos
        extremos?
    \item $S_{\min} = \{\{x\}, \{y\}, \{z\}\}, S_{\max} = \{\{x, y\}, \{x, z\},
        \{y, z\}\}$
  \end{itemize}
\end{frame}

\begin{frame}
  \frametitle{Subconjuntos de \emph{posets}}
  \begin{itemize}
    \item Em um \emph{poset} qualquer $(S, \preccurlyeq)$, a relação é
        preservada para todos os subconjuntos de $S$
    \item Elementos extremos para estes subconjuntos têm nomes especiais
    \begin{itemize}[itemsep=0pt]
      \item Tome um subconjunto $K \subseteq S$ e elemento $x \in S$ quaisquer
      \item $x$ está na \textbf{cota superior} de $K$ se $k \preccurlyeq x,
          \forall k \in K$, e na \textbf{cota inferior} de $K$ se $x
            \preccurlyeq k, \forall k \in K$
      \item Não existe notação definida para as cotas de $K$ em relação à $S$
    \end{itemize}
  \end{itemize}
\end{frame}

\begin{frame}
  \frametitle{Elementos extremos em cotas}
  \begin{itemize}
    \item Dentro da cota superior para um subconjunto de um \emph{poset},
        pode existir um elemento mínimo neste, e este é chamado de
          \textbf{menor cota superior}, ou ínfimo --- $\inf(K)$
    \item Analogamente, pode existir um elemento máximo na cota inferior,
        chamado de \textbf{maior cota inferior}, ou supremo --- $\sup(K)$
    \item Se $\inf(K)$ existe e pertence a $K$, então é um elemento
        minimal ou mínimo de $K$, e de maneira dual, $\sup(K)$ é um
          elemento maximal ou máximo de $K$
  \end{itemize}
\end{frame}

\begin{frame}
  \frametitle{Exemplo}
  \framesubtitle{Elementos extremos em cotas}
  \centering
  \begin{tikzpicture}
    \node [small, label=left:{\tiny  6}]  (6) at (-0.5, -1.6) {};
    \node [small, label=right:{\tiny 10}] (10) at (0.5, -1.6) {};
    \node [small,                      label=above:{\tiny 60}] (60) {};
    \node [small, below left  = of 60, label=left:{\tiny 12}] (12) {};
    \node [small, below       = of 60, label=right:{\tiny 20}] (20) {};
    \node [small, below right = of 60, label=right:{\tiny 30}] (30) {};
    \node [small,       left  = of  6, label= left:{\tiny  4}] (4) {};
    \node [small,       right = of 10, label=right:{\tiny 15}] (15) {};
    \node [small,       below = of 12, label=left:{\tiny  2}] (2) {};
    \node [small,       below = of 20, label=left:{\tiny  3}] (3) {};
    \node [small,       below = of 30, label=right:{\tiny  5}] (5) {};
    \node [small,       below = of  3, label=below:{\tiny  1}] (1) {};
    \draw  (2) edge  (4)  (2) edge  (6)  (2) edge (10)  (3) edge  (6)
           (3) edge (15)  (4) edge (12)  (4) edge (20)  (5) edge (10)
           (5) edge (15)  (6) edge (12)  (6) edge (30) (10) edge (20)
          (10) edge (30) (15) edge (30) (12) edge (60) (20) edge (60)
          (30) edge (60)  (1) edge  (2)  (1) edge  (3)  (1) edge  (5);
  \end{tikzpicture}
  \begin{itemize}
    \item Considere o \emph{poset} de divisores de $60$ e $K = \{6, 15\}$
    \begin{itemize}[itemsep=0pt]
      \item A cota superior é $\{30, 60\}$, a cota inferior é $\{1, 3\}$,
          $\inf(K) = 30$ e $\sup(K) = 3$
    \end{itemize}
  \end{itemize}
\end{frame}

\begin{frame}
  \frametitle{Exemplo}
  \framesubtitle{Elementos extremos em cotas}
  \centering
  \begin{tikzpicture}
      \node [small,                    label=above:{\tiny h}] (h) {};
      \node [small, right      = of h, label=above:{\tiny j}] (j) {};
      \node [small, below      = of j, label=right:{\tiny f}] (f) {};
      \node [small, below      = of f, label=right:{\tiny e}] (e) {};
      \node [small, below      = of e, label=right:{\tiny c}] (c) {};
      \node [small, below left = of h, label=left:{\tiny g}] (g) {};
      \node [small, below      = of g, label=left:{\tiny d}] (d) {};
      \node [small, below      = of d, label=left:{\tiny b}] (b) {};
      \node [small, below left = of c, label=below:{\tiny a}] (a) {};
      \draw (a) edge (b) (a) edge (c) (b) edge (d) (d) edge (g) (g) edge (h)
            (c) edge (e) (e) edge (f) (f) edge (j) (f) edge (h) (d) edge (f)
            (b) edge (e);
  \end{tikzpicture}
  \begin{itemize}
    \item Considere o \emph{poset} acima e $K = \{a, c, d, f\}$
    \begin{itemize}[itemsep=0pt]
      \item A cota superior é $\{f, h, j\}$, a cota inferior é $\{a\}$,
          $\inf(K) = f$ e $\sup(K) = a$
    \end{itemize}
  \end{itemize}
\end{frame}

\begin{frame}
  \frametitle{Material de estudo}
  \nocite{*}
  \bibliographystyle{plain}
  \bibliography{ref}
  \begin{itemize}
    \item Leitura das páginas 618-626 e resolução dos exercícios 1-2, 5-6,
        9-11, 13-16, 18, 20-25, 32-35
  \end{itemize}
\end{frame}

\end{document}
