\documentclass{article}

\usepackage[brazil]{babel}
\usepackage[T1]{fontenc}
\usepackage[utf8]{inputenc}
% \usepackage[a4paper, margin=2cm]{geometry}

\title{Parametrização de desempenho do esquema de
assinatura digital Winternitz e suas variantes}
\author{Novembro de 2017}
\date{}

\begin{document}

\maketitle

\section{Introdução}



\section{Motivação}

\section{Contribuição}

Algoritmos utilizados em esquemas de assinatura digital atualmente, como RSA
e ECDSA, têm sua segurança baseada em calcular a fatoração de números muito
grandes ou logaritmos discretos. Este tipo de cômputo pode ser realizado por
um computador quântico suficientemente poderoso, utilizando algoritmos já
conhecidos (e.g. algoritmo de Shor). Deste modo, para manter o ecossistema de
assinaturas digitais continuamente seguro, é necessário oferecer alternativas
pós-quânticas, ou seja, resistentes a computadores quânticos. Este trabalho
busca apresentar esquemas baseados apenas em funções de resumo
criptográficas, cuja segurança baseia-se apenas na resistência à colisão da
função escolhida, com o objetivo de mostrar que a construção de esquemas de
assinatura seguros independe de problemas considerados difíceis em teoria de
números ou álgebra, levando em conta apenas algoritmos quânticos, como o
algoritmo de Grover. Ademais, apresentam-se soluções para alguns problemas
deste tipo de esquema, como o tamanho e possibilidade de reutilização das
chaves pública e privada, assim como uma variada gama de algoritmos com estas
características, particularmente os baseados em árvores de Merkle e variantes
do esquema Winternitz.

Funções de resumo criptográficas são essenciais em diversas aplicações de
segurança da informação, como códigos de autenticação de mensagens (MACs),
identificação de arquivos a partir de uma `'impressão digital`' única, detecção
de perda de informações em uma transmissão volátil etc., e têm uso disseminado
em esquemas de assinatura digital, os quais são o foco deste trabalho. Uma
característica relevante destes é a possibilidade da demonstração matemática da
autenticidade de uma mensagem transmitida entre entidades, sejam estas
confiáveis ou não, utilizando as funções citadas anteriormente, para que seja
estabelecida uma comunicação segura.

A maior parte dos esquemas de assinatura digital, na prática, utilizam
criptossistemas de chave pública baseados em problemas de teoria de números --
fatoração de inteiros, logaritmo discreto -- que, atualmente, não podem ser
computados em tempo polinomial (ou seja, seu tempo de execução é limitado por
uma expressão polinomial relacionada com o tamanho da entrada do algoritmo).
Entretanto, utilizando-se de um computador quântico, tais problemas podem ser
resolvidos em tempo polinomial \cite{shor1999polynomial}.

A criptografia \emph{pós-quântica} encarrega-se de buscar algoritmos
criptográficos cuja complexidade independe de problemas futuramente
solucionáveis em teoria de números. Como visto em \cite{Merkle1990}, um
criptossistema de chaves públicas pode ser definido de forma independente da
função de resumo criptográfica utilizada, assim possibilitando a exploração de
diversas combinações, cujos elementos essenciais (e.g. árvores de Merkle) não
influenciam na segurança do esquema.

Neste trabalho, utilizamos funções de resumo criptográficas para a construção
de esquemas de assinatura digital, considerados seguros se e somente as funções
forem resistentes à colisão (a inviabilidade computacional de encontrar duas
mensagens distintas que, submetidas à mesma função, retornam a mesma saída),
tornando os requisitos de segurança mínimos para os esquemas.

\section{Objetivos}

\noindent \emph{Objetivo geral.} Apresentar um estudo detalhado sobre esquemas
de assinatura digital baseados em funções de resumo criptográficas, partindo de
esquemas de assinatura única \cite{Lamport1979}, observando o refinamento
destes, até o estado da arte, onde não é necessário saber quantas assinaturas
foram geradas anteriormente \cite{Bernstein2015}, bem como implementações em
linguagem de alto nível para a fácil compreensão destes esquemas. \\

\noindent \emph{Objetivos específicos.} Descrever os esquemas de assinatura
digital única Lamport-Diffie e Winternitz; descrever os esquemas de assinatura
digital baseado em árvores de Merkle -- \emph{Merkle Signature Scheme},
\emph{eXtended Merkle Signature Scheme}; implementar os esquemas supracitados;
comparar o desempenho destes algoritmos entre si, utilizando funções de resumo
criptográficas e parâmetros internos aos algoritmos distintos, onde aplicável.
\\

\noindent \emph{Escopo do trabalho.} Não se aplica ao conteúdo deste trabalho
a análise profunda de provas de segurança definidas por um modelo adversarial
teórico -- ou seja, demonstrar que um atacante deve resolver um problema muito
difícil para tornar o algoritmo inseguro, bem como algoritmos de criptografia
pós-quânticos baseados em outras estruturas matemáticas (reticulados, teoria
de códigos etc.) ou algoritmos clássicos (RSA, DSA, ECDSA etc.). \\

\noindent \emph{Critérios de aceitação.} Estudo e implementação de pelo menos
dois esquemas de assinatura digital única (Lamport-Diffie, Winternitz),
a estrutura de dados chamada de árvore de Merkle e um esquema de assinatura
digital composto da união destes elementos, como o XMSS \cite{Buchmann2011}. \\

O trabalho será desenvolvido utilizando a infraestrutura e recursos do
Laboratório de Segurança em Computação (LabSEC/UFSC), onde será estudada
bibliografia referente aos temas abordados nesta pesquisa buscando encontrar
as vantagens e desvantagens entre cada um dos esquemas de assinatura digital
escolhidos, bem como observar seu desempenho e tamanho de elementos como
par de chaves e assinatura, ao utilizar funções de resumo criptográficas
distintas em implementações produzidas ou fornecidas.

\bibliographystyle{plain}
\bibliography{main}

\end{document}