\documentclass[12pt]{article}

\usepackage[brazil]{babel}
\usepackage[T1]{fontenc}
\usepackage[utf8]{inputenc}
\usepackage[a4paper, margin=3cm]{geometry}

\usepackage{parskip}

\title{Parametrização de desempenho do esquema de
assinatura digital Winternitz e suas variantes}
\author{Novembro de 2017}
\date{}

\begin{document}

\maketitle

\section{Introdução}

A utilização de protocolos criptográficos entre dispositivos e entidades é
amplamente disseminada e considerada um fator crítico no contexto da validação
de quaisquer atos de comunicação realizados por estes indivíduos, em virtude da
possível criticalidade e sensibilidade atribuídas aos dados transmitidos.
Esquemas de assinatura digital são comumente utilizados para assegurar, de
maneira formal \cite{Goldreich:2004:FCV:975541}, esta validação através da
autenticidade e não-repúdio do remetente e certeza da integridade dos dados em
um contexto, a fim de traduzir o resguardo provido por uma assinatura de
próprio punho no mundo real.

Na prática, a maior parte destes esquemas utilizam criptossistemas de chave
pública baseados em problemas da teoria dos números, como a fatoração de
inteiros ou resolução do logaritmo discreto para números grandes (e.g. RSA
\cite{Rivest:1978:MOD:359340.359342} e ECDSA \cite{Johnson2001}). Este fato
provê a segurança necessária para os esquemas em computadores clássicos
(mecânicos) em virtude da não-existência de algoritmos de tempo polinomial para
a resolução destes problemas. Entretanto, em computadores quânticos, algoritmos
dessa forma já existem -- em especial, o algoritmo de Shor
\cite{Shor:1997:PAP:264393.264406} -- efetivamente tornando estes esquemas
inseguros neste novo contexto. 

Para combater esta situação, a criptografia pós-quântica encarrega-se de buscar
algoritmos criptográficos cuja segurança é considerada razoável, mesmo
utilizando-se de um computador quântico e algoritmos especializados para o
ataque. Uma definição viável de esquema de assinatura digital resistente a este
tipo de computador pode ser dada apenas com funções de resumo criptográfico,
construídas a partir de funções de mão única \cite{cryptoeprint:2005:328}. De
fato, estas funções, desde que apresentem requisitos de segurança como
resistência à segunda pré-imagem e à colisões, são necessárias e suficientes
para a construção de esquemas bem comportados
\cite{Rompel:1990:OFN:100216.100269}.

Esquemas de assinatura digital baseados em funções de resumo criptográfico
consistem da combinação de um esquema de assinatura digital única, onde apenas
uma mensagem pode ser assinada de modo seguro, e uma estrutura chamada de
árvore de Merkle \cite{Merkle:1989:CDS:118209.118230}, que abriga diversos
pares de chave únicos como suas folhas, e reduz a verificação destes para uma
única chave, codificada em sua raiz. Esta árvore é construída com a
concatenação de resumos criptográficos do conteúdo dos nós, habilitando assim a
assinatura de diversas mensagens. Como uma função específica não é necessária,
é possível obter uma grande variedade de esquemas, garantindo a versatilidade
desta abordagem para utilização de assinaturas digitais.

Embora os esquemas iniciais tenham sido construídos sem atenção particular à
eficiência de modo geral (e.g. o esquema de assinatura única de Lamport-Diffie
\cite{Lamport1979} assina apenas um \emph{bit} de informação em sua forma mais
simples), muitos resultados práticos habilitam a redução contínua do tempo de
verificação da assinatura, tamanho e tempo para geração do par de chaves e
assinatura, bem como avanços teóricos possibilitam a utilização de funções com
requisitos de segurança mínimos, garantem o conceito de sigilo encaminhado
\cite{Buchmann:2011:XPF:2184003.2184011} (comprometimento de uma chave não
implica na segurança de mensagens que utilizaram esta chave anteriormente) e da
ausência de estado \cite{Bernstein2015} (o esquema não necessita registrar
quais chaves já foram utilizadas, e.g. marcando as folhas da árvore de Merkle).

\section{Motivação}

O esquema de assinatura digital única Winternitz é proposto como uma
generalização de \cite{Lamport1979}, permitindo a assinatura de múltiplos
\emph{bits} simultaneamente. Tal comportamento é configurado por um parâmetro
$w$, que implica imediatamente no tamanho do par de chaves e na velocidade que
a assinatura é criada e/ou verificada. Com um parâmetro $w$ bem escolhido, o
esquema torna-se relativamente eficiente, e a valer, este esquema, e sua
variante WOTS+ \cite{cryptoeprint:2017:965} cujos requisitos de segurança são
diminuídos, têm sido utilizados amplamente como parte de diversos esquemas
baseados em funções de resumo criptográfico, como SPHINCS \cite{Bernstein2015},
MSS \cite{Merkle:1989:CDS:118209.118230}, CMSS \cite{Buchmann2006}, GMSS
\cite{Buchmann2007} e XMSS \cite{Buchmann:2011:XPF:2184003.2184011}.

Como descrito em \cite{Bernstein:2008:PQC:1522375}, o algoritmo consiste na
aplicação da função de resumo criptográfico escolhida repetidamente sobre
blocos da chave privada, i.e. um encadeamento da função. O número de blocos
é decidido pelo parâmetro $w$ e representará quantos bits serão assinados
simultaneamente, bem como o tamanho máximo da cadeia de resumos para um
bloco para realizar a assinatura de cada bloco, é preciso determinar a
quantidade de vezes que o encadeamento será repetido. Isto pode ser feito
obtendo uma fração de $w$ \emph{bits} do resumo criptográfico da mensagem
que se deseja assinar. Uma assinatura, portanto, é um conjunto de cadeias
de resumos criptográficos da chave privada.

Winternitz figura como o esquema escolhido para assinar as folhas das
árvores de grande parte da literatura prática em cima de esquemas baseados
nesta estrutura de dados. Por exemplo, no caso do SPHINCS, construído em
cima do conceito de camadas de árvores de Merkle (a fim de descartar a
necessidade de manter o estado do esquema de assinatura digital), WOTS+ é
utilizado para a autenticação entre estas. Afirma-se, então, que é um
dos esquemas de assinatura única mais populares, tanto em resultados
práticos como teóricos -- denotando um interesse da comunidade acadêmica
por resultados que impliquem em melhoras no algoritmo.

\section{Contribuições propostas}

Devido à construção do algoritmo, e como os processos de geração de chaves,
assinatura e verificação são realizados, demasiadas aplicações da função
de resumo criptográfico escolhida são realizadas na vida útil de uma
assinatura digital. Assim, uma possível fronte de pesquisa consiste na
utilização ótima de recursos computacionais para que o desempenho de
Winternitz, e consequentemente de outros esquemas baseados em funções de
resumo criptográfico, seja melhorado.

Ademais, como a saída de uma função de resumo criptográfico deve conter,
em média, o mesmo número de 0s e 1s, isso implica que metade do processo
de encadeamento estará localizado na parte da assinatura, e a outra
metade na verificação. Entretanto, em situações onde assinaturas são
raramente produzidas mas frequentemente verificadas, e vice-versa, esta
distribuição de cômputos torna o processo não otimizado, de modo a
sugerir uma possível alteração no comportamento do algoritmo a fim de
considerar estes diversos casos de uso.

Isto implica na introdução de técnicas determinísticas a fim de
modificar o processo de encadeamento de resumos. O trabalho de
\cite{Steinwandt:2008:OSU:1412758.1412979} propõe um esquema de
assinatura digital baseado em Winternitz utilizando compressão de
cadeias repetidas (\emph{run-length encoding}) que resulta na
diminuição em $33\%$ do tempo de verificação para uma assinatura.
Diferentemente deste esquema, propõe-se a criação de passos
adicionais ao Winternitz convencional, na etapa de geração da
assinatura, a fim de customizar o tamanho da cadeia de resumos
e obter um resultado paramétrico à situação desejada.

\bibliographystyle{alpha}
\bibliography{main}

\end{document}