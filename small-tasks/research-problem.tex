\documentclass[12pt]{article}

\usepackage[T1]{fontenc}
\usepackage[a4paper, margin=1.5cm]{geometry}
\usepackage[colorlinks, urlcolor=blue, citecolor=red]{hyperref}
\usepackage[utf8]{inputenc}
\usepackage{amsmath, amsfonts, enumitem, parskip}

\newcommand{\hh}{\mathcal{H}}
\newcommand{\pk}{\mathcal{P}_{k}}
\newcommand{\sk}{\mathcal{S}_{k}}
\newcommand{\binwds}[1]{\{0, 1\}^{#1}}
\newcommand{\hash}[2][]{\mathcal{H}^{#1} (#2)}
\newcommand{\wots}{\textsc{Wots}}

\begin{document}

\textsc{Graduate Program in Computer Science,
  Universidade Federal de Santa Catarina} \\
\textsc{INE410111 (Research Methodology in Computer Science)}

\textsc{Gustavo Zambonin} \\
\textsc{Optimizing the Winternitz Digital Signature Scheme with
Message Randomizers} \\

Digital signature schemes are widely used in situations requiring secure and
tamper-free communication, some of which are software distribution and
communications over computer networks. Predominantly used are schemes in which
the underlying security argument is derived from computational hardness
assumptions of problems in number theory, such as integer factorization
discrete logarithm. However, with the advent of quantum computers, these
problems can be solved more efficiently with the aid of Shor's
algorithm~\cite{Shor:article:1997:oct}. Ergo, such schemes will become fragile
and may no longer guarantee properties required for digital signatures. The
alternative is to use post-quantum cryptography, that is, algorithms that are
thought to be resistant against such techniques.

There are various strategies proposed in the literature designed to produce
quantum-resilient digital signatures, making use of concepts such as coding
theory, lattice structures, multivariate polynomials over finite fields and
cryptographic hash functions. In special, for the latter option, the robust
flexibility and abundant availability of these functions in all kinds of
computational systems make hash-based digital signature schemes potential
candidates to replace traditional schemes. Additionally, it has been shown that
one-way functions (the theoretical foundations of a hash function) are the only
constructions needed to build a secure digital signature
scheme~\cite{Rompel:inproc:1990:may}. These reasons are strong indicators that
hash-based signature schemes are presumably safe, and extremely fast in
practice.

The idea of using only hashes to sign digital messages is not new. Lamport
proposed~\cite{Lamport:report:1979:oct} a very simple and secure scheme to
generate and verify digital signatures by the end of the seventies. It
considers a pair of pseudorandom words as the private key and their hashes as
the public key. The signer signs a single bit of information by choosing to
distribute one string from the private key, on whether the bit is 0 or 1, as
the signature. The verifier then computes the hash of this string, comparing it
with part of the public key. By extending this idea to sign messages of
arbitrary sizes, some drawbacks emerge, such as large sizes for key pairs and
signatures.

These obstacles have been the subject of research and several improvements have
been proposed since the publication of the initial design. Perhaps the most
promising method is the Winternitz one time signature scheme (\wots{})
due to Merkle~\cite{Merkle:inproc:1989:aug}. This scheme is a generalization of
the Lamport scheme. Unlike its predecessor, \wots{} allows more than one
bit to be signed simultaneously through the parameter $w$, decreasing the size
of key pairs and signatures. In fact, the lower cost of the key has been
replaced with a higher processing cost, since the total number of hashes to
sign a document and verify a signature depends on this parameter.

Various hash-based signature schemes have been proposed based on, or using
\wots{} as part of the scheme. By way of illustration, organizations such
as the U.S. National Institute of Standards and Technology (NIST) and the
Internet Research Task Force (IRTF) have shown great interest in the
research and standardization of these algorithms. The SPHINCS+
proposal~\cite{Bernstein:misc:2017:dec}, a variant
of~\cite{Bernstein:inproc:2015:apr} that reduces the signature size and overall
execution time, was submitted to NIST as a candidate for this process. Besides,
there has been continuous activity on the development of informational
documents from the IRTF~\cite{Huelsing:report:2018:may,McGrew:report:2018:apr},
featuring variant schemes derived from \wots{}.

Observe that the aforementioned costs for signature generation and verification
are not distributed evenly, since they are derived from the result of the
application of the fingerprint function to a message. Hence, these can be
tailored to a faster signature generation or verification by carefully choosing
a message derived from the original. We believe that the cost difference
between these steps is not exploited to fit specific needs. Often, signatures
are generated only once on a targeted device, which can be properly configured
to accommodate the required resources. However, the resources available for the
signature verification are unlikely to be predictable in advance, since this
step can be performed by any other kind of devices. For this reason, in
practice, it is common to choose parameters that make it easier for a third
party to verify signatures (\emph{e.g.} the use of the public exponent
$65{,}537$ of the RSA signature scheme). Conversely, there may exist devices
with constraints for signature generation.

This situation is no different with hash-based signature schemes. Several
papers proposed in the literature aim to improve the performance of the
verification step for \wots{}, such as~\cite{Cruz:inproc:2016:oct,
McGrew:report:2018:apr,Steinwandt:article:2008:oct}.
While these works modify the structure of the Winternitz scheme, our main
proposal consists of a pre-processing of the message, keeping the scheme
intact. Another advantage of our scheme is the possibility of using larger
values for the parameter $w$, substantially reducing the size of the
\wots{} signature. Furthermore, these proposals focus only on the
optimization of the signature verification step, whereas our method also works
for hastening the creation of signatures.

We propose two methods to reduce costs for the signature generation or
verification steps in \wots{}. Our first method changes the checksum
computation slightly, by padding unused bits with ones in the place of zeros.
This can only be used to improve signature verification run times, while
padding with zeros is already optimal for signature generation. Our second
and main approach is based on~\cite{Steinwandt:article:2008:oct}, where we
append a cryptographic nonce to a document to be signed and hash it. We repeat
the process and search for hash outputs that can optimize signature generation
or verification. In fact, by doing this, we do not change the underlying
\wots{} algorithm.  We propose a fixed amount of operations before the
signature is generated, so that the chosen step can be computed much faster.
This is actually a trade-off choice, where improving one step results in more
computations for the other. However, there is always an additional fixed cost
to the signature generation. Both proposals can be applied independently or
together for better results.

\bibliographystyle{alpha}
{\footnotesize
\bibliography{\jobname}}

\end{document}
