\documentclass[12pt]{article}

\usepackage[T1]{fontenc}
\usepackage[a4paper, margin=1.5cm]{geometry}
\usepackage[colorlinks, urlcolor=blue, citecolor=red]{hyperref}
\usepackage[utf8]{inputenc}
\usepackage{amsmath, amsfonts, enumitem, parskip}

\newcommand{\hh}{\mathcal{H}}
\newcommand{\pk}{\mathcal{P}_{k}}
\newcommand{\sk}{\mathcal{S}_{k}}
\newcommand{\binwds}[1]{\{0, 1\}^{#1}}
\newcommand{\hash}[2][]{\mathcal{H}^{#1} (#2)}

\begin{document}

\textsc{Graduate Program in Computer Science,
  Universidade Federal de Santa Catarina} \\
\textsc{INE410135 (Blockchain and Cryptocurrencies Technologies)}

\textsc{Notes on Coursera's ``Bitcoin and Cryptocurrency Technologies''
  1st Lecture} \\
\textsc{Gustavo Zambonin}

\subsection*{Introduction}

Currencies of any kind, even in their most primitive forms, need some sense of
security to achieve pertinent ownership of valuable goods, and enable users to
exchange these accordingly. This concept is generally represented by means of
physical money, such as coins, notes and checks, and oversight by a bank.
However, transactions nowadays are frequently done in the virtual medium,
executed through ATMs, payment cards, personal computers and other devices with
digital components. It could be argued that this paradigm shift facilitates the
migration of entire currency infrastructures to entirely digital approaches.
Yet, various issues arise when taking into account the security of said systems
with respect to decentralization of trust, anonymity of transactions, storage
of digital coins and various other problems that touch economic and social
disciplines.

To efficiently address these situations, a \emph{cryptocurrency} may be
established through the use of cryptography to allow secure transactions and
creation of new currency units. By design, dissociation from central banking is
given special emphasis, therefore making these currencies inherently
unregulated.  Nevertheless, there has been considerable adhesion from
individuals and businesses alike. Ergo, it is reasonable to understand how
concepts such as cryptographic hash functions, digital signatures, data
structures, distributed ledgers and merge these to construct simple examples of
cryptocurrencies.

\subsection*{Cryptographic hash functions}

A hash function $\hh{}$ deterministically maps values from a set with possibly
infinite size to another, strictly smaller finite set. Hash functions with added
properties that make them suitable for use in security applications are called
cryptographic hash functions; these can provide assurance of data integrity,
even if the data is stored in an insecure place. Outputs of these functions are
generally called digests. For any $\hh{}$ to be considered cryptographic,
solving the three problems listed below should be computationally infeasible:

\begin{enumerate}[label= (\roman*)]
  \item Given a digest $h$, find the original message $m$ such that
    $\hash{m} = h$. $\hh{}$ is said to be \emph{preimage} resistant if
    this cannot be solved efficiently.
  \item Given a message $m_0$, find a message $m_1$ such that $m_0 \neq m_1$
    and $\hash{m_0} = \hash{m_1}$. $\hh{}$ is said to be \emph{second preimage}
    resistant if this cannot be solved efficiently.
  \item Given messages $m_0$ and $m_1$ such that $m_0 \neq m_1$, then
    $\hash{m_0} = \hash{m_1}$. $\hh{}$ is said to be \emph{collision} resistant
    if this cannot be solved efficiently.
\end{enumerate}

Note that (ii) and (iii) present a slight difference: in the former, an
adversary may not choose the first message, whereas in the latter, it is free
to choose any pair of messages. It is also desirable for $\hh{}$ that its
mappings happen in such a way that digests have no apparent relation between
themselves and their preimages. Furthermore, another suitable characteristic is
the avalanche effect, based on the concept of diffusion, stating that any
changes in a message $m$ should result in a totally different resulting digest.

\subsection*{Digital signatures}

A digital signature scheme is a mathematical construction that enables certain
properties to be derived from signed messages. Specifically, it uniquely
identifies and therefore \emph{authenticates} the signer; it preserves the
\emph{integrity} of messages, confirming that these were not modified in
transit; and it prevents denial from the signer that the message was signed and
sent --- \emph{non-repudiation}.

These schemes are based on public key cryptography, in which an entity has
possession of a key pair, used in a way such that a document signed with one of
the keys may only be verified using its correspondent in the pair. Evidently,
to prevent impersonation, one of the keys is kept private, while the other is
published to enable verification by any party. Digital signature schemes are
defined as a triple of algorithms, listed below.

\begin{enumerate}[label= (\roman*)]
  \item Key pair generation \textsc{Gen}, that takes as input a security
    parameter $1^{n}, n \in \mathbb{N}^{*}$ and outputs a random key pair
    $(\sk{}, \pk{})$.
  \item Signature generation \textsc{Sig}, that takes as input a message $m$
    and the secret key $\sk{}$, and outputs a signature $\sigma$. A common
    practice is to first hash the message then sign the resulting digest.
  \item Signature verification \textsc{Ver}, that takes as input $m$, $\sigma$
    and the public key $\pk{}$, and outputs either $0$ for a failed
    verification or $1$ for a correct one.
\end{enumerate}

These algorithms are constructed such that signatures on any possible messages
are verifiable using any generated keys, that is,
$\forall (p, s) \in \textsc{Gen}(1^{n}), \forall w \in \binwds{*},
\text{Pr}[\textsc{Ver}(p, w, \textsc{Sig}(s, w)) = 1] = 1$. However, being in
accordance to this description is not a sufficient condition for a scheme to
be secure, only correct. This notion is achieved in its strongest form when an
adversary can forge a valid signature using any message.

\subsection*{Distributed ledgers}

A ledger summarizes financial information in the form of debits and credits,
showing balances for certain periods of time, allowing visualization and
management of transactions. To achieve decentralization, the ledger must be
replicated and synchronized across heterogeneous nodes, generally through peer
to peer networks. Design of distributed ledgers generally come in the form of
graphs, in which nodes containing transaction data are connected by pointers to
hashes of other transactions, either linearly (in the case of blockchain), in
the form of a directed acyclic graph, or using a binary hash tree, called a
Merkle tree. Hence, to falsify a transaction, an adversary would have to forge
every transaction before it.

\subsection*{Examples of cryptocurrencies}

Summarizing these concepts into an example may be done through the presentation
of GoofyCoin. Goofy may create coins whenever needed, and this transaction
shall be signed by its private key. Available currency may then be transferred
to any recipient with a transaction signed by the coin's owner. Clearly, these
transactions are linked through hash pointers. However, this design does not
prevent the \emph{double-spending problem}, in which an entity may create
transfer transactions haphazardly. To be secure, this cryptocurrency could make
use of the blockchain design, that takes in account the history of
transactions. Hence, transactions are only valid if coins were really created,
were not already consumed, the value being received is equal to the value being
given, and shall be signed by all owners of the coins being transacted.

\end{document}
